\documentclass[preprint]{aastex} 

\usepackage[top=1in, bottom=1in, left=1in, right=1in]{geometry}
\usepackage{amsmath}
\usepackage{graphicx}
\usepackage{mdwlist}
\usepackage{natbib}
\usepackage{natbibspacing}
\setlength{\bibspacing}{0pt}
\setlength{\parskip}{0pt}
\setlength{\parsep}{0pt}
\setlength{\headsep}{0pt}  
\setlength{\topskip}{0pt}
\setlength{\topmargin}{0pt}
\setlength{\topsep}{0pt}
\setlength{\partopsep}{0pt}
\setlength{\footnotesep}{8pt}
\pagestyle{empty}
\citestyle{aa}

\newcommand{\simgt}{\stackrel{>}{_{\sim}}}
\def\kperp{k_{\bot}}
\def\kpar{k_{\|}}
\def\k{{\bf k}}
\def\sky{{\theta}}
\def\HI{{H{\small I }}}
\def\HII{{H{\small II }}}
\def\xHI{{x_{\rm\HI}}}

%\usepackage{subfig}
%\usepackage[countmax]{subfloat}

\begin{document}
\title{Data Management Plan}

% Plans for data management and sharing of the products of research. Proposals must include a supplementary document of no more than two pages labeled “Data Management Plan”. This supplement should describe how the proposal will conform to NSF policy on the dissemination and sharing of research results (see AAG Chapter VI.D.4), and may include:
% 
%     the types of data, samples, physical collections, software, curriculum materials, and other materials to be produced in the course of the project;
% 
%     the standards to be used for data and metadata format and content (where existing standards are absent or deemed inadequate, this should be documented along with any proposed solutions or remedies);
% 
%     policies for access and sharing including provisions for appropriate protection of privacy, confidentiality, security, intellectual property, or other rights or requirements;
% 
%     policies and provisions for re-use, re-distribution, and the production of derivatives; and
% 
%     plans for archiving data, samples, and other research products, and for preservation of access to them.



% The calibration, imaging, and data-reduction pipelines associated with
% evaluating the performance of the 7-element hexagonal close-packed array
% will be
% implemented using the open-source AIPY software framework, and will run on the
% computing cluster and data archive for PAPER
% that is maintained at UPenn.  This currently consists of 22 nodes connected by Gb
% ethernet. These nodes consist of 16 Dell PowerEdge 1950s (dual 4-core Intel Xeon L5420 \@ 2.50GHz)
% and 6 Dell PowerEdge R410 (32 GB RAM; dual 6-core Intel Xeon E5649 \@ 2.53GHz),
% for a total of 200 cores and 448 GB RAM.  Fast working data storage is provided
% by a network storage system (NSS) based around two Dell MD1200s, with a total
% RAID storage capacity of 140 TB.
% This cluster
% will also host the principle storage of data, and data distribution will
% be managed through account access to this system.  All observed data
% will be stored and archived indefinitely at this location, with access
% provided to collaborators upon request.
% 
% Final antenna element construction schematics will be made available
% for future production of the HERA elements. These designs will be in industry standard format.
% Data collected about the electromagnetic properties of the small array will be
% disseminated in a scientific publication characterizing in detail the
% performance of the HERA elements.

\parskip 0.25in

The HERA measurements will be released to the community after an 18-month proprietary period and hosted on a public server at MIT (see \S\ref{sec:DataProducts}).  A large number of data products will be available.  Imaging products will include coarse real-time calibrated snapshot images in addition to high dynamic range survey maps from the FHD pipeline.  Calibrated, compressed, LST-binned visibilities will also be made public for re-analysis.  Higher cadence data sets will not be hosted publicly in a continuous fashion due to data transfer limitations, but will be available to any users upon request.  A number of derived data products will also be made accessible.  Source catalogs will be provided, and a Global Sky Model of diffuse foregrounds will be updated with new observations and released.  Foreground-subtracted data cubes of high-redshift $21\,\textrm{cm}$ intensity maps and the associated ionization field will be available for cross-correlation studies, as well as to provide context for high redshift optical and IR studies.


This proposal will involve the collection and analysis data from the HERA-61 instrument.
Data reduction will use both internally generated custom software (e.g. AIPY\footnote{\tt https://pypi.python.org/pypi/aipy}) and publicly available packages (e.g. CASA\footnote{ \tt http://casa.nrao.edu/}).  The plan for data transfer from South Africa and long-term storage in the US, together with the compression, validation, and reduction software are heavily leveraged from existing experience with PAPER. \parskip 0pt

Briefly, data generated by the HERA correlator is transferred to a RAID system, and compressed on-site using a small cluster.  Compressed data is streamed over the internet back to Penn, as well as being transferred to small disks which are mailed back to the US.  Long term storage is effected at Penn on co-PI Aguirre's cluster and network storage (see Facilities, Equipment, and Other Resources).  Further data analysis and reduction are performed on this cluster, and subsets of the data distributed to collaborators as needed.  These computing resources, while shared, are adequate for the needs of this grant, and are currently handling a data volume approximately four times larger than HERA-61 will produce.  

The data reduction will be described in publication in refereed articles.  Other
scientists who may wish to access the data may ask the collaboration for access once the quality of the data is assured.  The Penn cluster will hold a compressed archive in {\sc miriad} format which will be accessible via the Internet 18 months after the data is processed. Individual inquiries for other data or custom software will be honored on a best-effort basis. 

\end{document}
