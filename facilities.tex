\documentclass[11pt]{article}
\usepackage{fullpage}

%Facilities, Equipment, and Other Resources: In order for NSF, and its
%reviewers, to assess the scope of a proposed project, all organizational
%resources necessary for, and available to a project, must be described in this
%section of the proposal (see GPG Chapter II.C.2.i for further information).
%Proposers should describe only those resources that are directly applicable.
%Proposers should include an aggregated description of the internal and external
%resources (both physical and personnel) that the organization and its
%collaborators will provide to the project, should it be funded. Such
%information must be provided in this section, in lieu of other parts of the
%proposal (e.g., budget justification, project description). The description
%should be narrative in nature and must not include any quantifiable financial
%information.

\begin{document}
\pagestyle{empty}

\section*{Facilities, Equipment, Other}

The work described in this proposal will take place at eight locations, as discussed below. 

\subsection*{\it Arizona State University}

Several ASU low-frequency developmental efforts will contribute directly to the
HERA effort. ASU will support the deployment of an antenna beam mapping
octocopter system which is currently undergoing testing for MWA, PAPER and
other low frequency experiments. This system will enable the precise mapping of
the antenna response as a function of frequency. Several developments from the
ongoing Experiment to Detect the Global EoR Step (EDGES, supported by NSF grant
\#1207761) will also be directly applicable.  Most notable is an ongoing effort
to improve the precision of low-frequency antenna response calibration by two
orders of magnitude. The resources at the disposal of the ASU low-frequency laboratory
include a 5000 square foot dedicated electronics lab complete with vector
network analyzers, FPGA development stations, high-accuracy noise sources,
outdoor antenna test facility, 6-axis micromill and several highly qualified
technicians.

Adjacent to the Low-Frequency laboratory is the Laboratory for Astronomical and Space
Instrumentation (LASI) which assembles and tests space instrumentation.
Proximity to the ASU Machine Shop (which has produced space-qualified hardware
for Mars missions) offers quick custom machining and integration of pieces in
instrument assembly. 

The ASU High Performance Computing Initiative maintains a 5000-node cluster
(Saguaro) with 11 TB of aggregate RAM and 215 TB of high-speed disk space that
is available to all students and researchers through an internal time
allocation process. A cluster dedicated to low-frequency astronomy, consisting
of three high-power (16 core, 128G of RAM, 75TB of storage) nodes suitable for
algorithm development and data exchange, is maintained by the co-I and
available to all collaborators.

\subsection*{\it Massachusetts Institute of Technology}

As part of the MIT's group participation in the MWA and with support from an
MRI grant, a dedicated cluster of PC-Linux computers and storage are being
purchased and installed in a phased manner as they are needed to accommodate
MWA data. The final configuration will consist of $\sim$15 computers and $>$1.0 Pbytes
of storage space. The computers will typically have 12 cores and 100 GB of RAM,
providing enough compute power to produce the data cubes required for EoR
science as the data arrive at MIT. The MWA data-taking will be complete by May
2015, and the cluster processing capacity will be made available to HERA in
September 2015, when it is anticipated that the standard pipeline processing of MWA 
data will be complete.

At MIT, we have developed a 21 cm power spectrum estimation pipeline using
quadratic estimator and Fisher matrix formalism to optimally measure power
spectra and rigorously keep track of estimator errors and correlations.  The
technique was adapted (Liu \& Tegmark 2011) from earlier work on the CMB and
galaxy surveys, accelerated to run in $\mathcal{O}(N\log N)$ by exploiting
various symmetries (Dillon et al. 2013a), and applied successfully to MWA
prototype data (Dillon et al. 2013b). The statistical technology and code is
straightforwardly adaptable to analyzing HERA data, and is furnished to the project.

The hexagonal layout of the HERA array core allows unprecedented opportunities
for automated precision calibration and quality control using massive baseline
redundancy (Liu et al 2010). This approach was successfully prototyped in the
MITEoR experiment, and the resulting hardware and software is now available for
HERA use. This includes the MITEoR 64 dual-polarization correlator, including 4
ADCx64 boards, 4 ROACH-II processing boards, 8 ROACH-I processing boards and
two data acquisition servers, as well as 64 dual-polarization MITEoR elements
with analog signal chains, including nodes, Walsh modulation boards, and
receivers.  The use of this system, as well as the underlying technology
and software libraries, are being furnished to the HERA project.

The massive baseline redundancy of HERA also provides a potential for
significant efficiency improvements to the correlator though 2D FFT's on the
hexagonal antenna lattice, cutting the numerical load scaling of the X-engine
from $\mathcal{O}(N^2)$ to $\mathcal{O}(N \log N)$. This would free up a significant fraction of the GPU
resources, which could instead be used for automated real-time calibration and
quality control.


\subsection*{\it NRAO} 

NRAO contributes support for Bradley's and Carilli's involvement in the
project, and provides access to the NRAO site near Green Bank, WV,
including the infrastructure and systems associated with the PAPER-32
array deployed there, for the purpose of development and field studies
of the HERA system.  NRAO also contributes support for adding low-frequency
functionality to the CASA software package.  This software is publicly available,
and will directly benefit HERA's foreground imaging initiatives.

NRAO provides access to the PAPER field station located on the NRAO Green Bank, WV observatory
site. This field station houses a work area and plenty of space to deploy additional instrumentation.
Positioned within the National Radio Quiet Zone, this site is an ideal venue
for sensitive radio-frequency measurements due to its remoteness from
large urban areas and its exceptional laboratory infrastructure. A relatively
large amount of outdoor space is available adjacent to the station for the
deployment of antennas under test. A satellite downlink measurement system is
here, and the PAPER array is located nearby. Electromagnetic enclosures are
used to house all instrumentation to prevent self-interference as well as
interference with Observatory telescopes, in strict adherence to the radio
emissions control policy of the Observatory’s Interference Protection Group
(IPG).

PAPER equipment deployed at Green Bank will be used for antenna testing,
absolute-calibration balun development,
feed development, and other preliminary on-antenna testing activities.
The deployed system in Green Bank consists of 32 dual-polarization PAPER antennas,
with associated analog electronics, cables, and power, and a 32-antenna correlator.
This correlator consists of 16 quad-ADC boards, 8 IBOB processing boards,
4 ROACH processing boards, and a data-acquisition server.  Network infrastructure 
on site permits the direct transfer of data products to the computing server 
at UPenn.

A 800 sq. ft. research laboratory is located at the NRAO’s Technology Center.
It is equipped with two complete micro-assembly workstations for component
fabrication and rework as well as a suite of basic instruments necessary for
evaluation and repair. In addition, the laboratory contains several signal
generators, power meters, spectrum analyzers, an impedance bridge, network
analyzers, and noise figure meters for radio-frequency applications. A small
anechoic chamber is available for antenna impedance measurements, and an
computer-controlled environmental chamber is ready for studies of thermal- and
humidity-induced effects on component and subsystem designs. Feed and balun
development will routinely make
use of the NRAO machine shop facilities.

For feed development, NRAO provides access to the latest computer-aided design software. Autocad
Inventor is used for three-dimensional mechanical drawings. Agilent’s Advanced
Design System (ADS) is available for circuit and subsystem modeling while CST
Microwave Studio is used for electromagnetic simulations of antenna and other
RF structures. In addition, CF Design is available for modeling thermal
conduction and radiation pathways.

\subsection*{\it South Africa}

The recent and manifold infrastructure developments in South Africa are of
direct benefit to the project.  These relate to the radio-quiet observatory in
the Karoo area of Northern Cape, South Africa, where PAPER has been observing
over the last several years, as well as the PAPER electronics and support
infrastructure directly.

The PAPER equipment will be used directly in this project up through the
127-element design.  This allows use of a thoroughly commissioned system for
science observations until there is the need for the larger correlator.  The
PAPER 128 dual-polarization correlator includes 16 ADCx16 boards,  8 ROACH-II
processing boards, 8 servers each hosting 2 GTX690 dual-GPU cards, and two data
acquisition servers.  The data compression and storage system includes 5
computing servers and a 220 TB raid storage system.  In addition, the 128 PAPER
elements, with analog signal chains, nodes, and receivers will be used as
part of the first observations in an imaging configuration. The 12-meter
container, 120 dB shielded EMC enclosure housing the correlator and electronics
along with the power infrastructure (transformers, reticulation, and UPS) and
HVAC system will also all be used.

South Africa will relocate the PAPER equipment to a new nearby site that had previously
been set aside for SKA-Low.  This will include site clearing, road access,
power and internet infrastructure and service.  As part of the observatory
development a new shielded building (the Karoo Array
Processing Building) is currently well under
construction, and HERA will have use of this facility, which will be connected
to the HERA site by a dedicated fiber cable.  This facility has a fast internet
connection back to the Cape Town headquarters.

The South African group consists of a growing pool of experienced scientists in
this field.  Key among them is Bernardi, who will participate extensively in
this project.  South Africa also provides strong liaison support between the
groups, building on the excellent experience PAPER has enjoyed.  South African
researchers will also provide mirror support for the student exchanges described
in the Broader Impacts section of the Project Description.

In addition, SKA-SA assists with the unloading, assembly, and storage of
shipping crates. Two on-site mobile truck cranes are available, along with use
of a 30m x 18m dish construction shed that is shared with the KAT-7 construction.
Accommodation is provided for on-site PAPER visitors. Off-road pickup trucks
are available for use driving to and from site, as well as while on site.  Site staff are available for
occasional operations maintenance support.

A digital electronic laboratory is available in the Cape Town SKA-SA offices,
complete with hot-air rework stations and test equipment (bench
power supplies, signal generators, network analyzers, spectrum analyzers,
oscilloscopes up to 100GHz etc). There is an off-site RFI measurement facility
near Cape Town (Houwteq) replete with 
an anechoic
chamber, a large hangar shed, an open-area test site, and a vibration room for testing
transmitters and receivers.

\subsection*{\it Cavendish Astrophysics, Cambridge}

Cavendish Astrophysics will provide scientific and technical staff to assist the project.  This will consist of
support for a PhD student to explore imaging/calibration using standard software,
support for a postdoc to explore Bayesian power spectral analysis and parameter estimation,
partial support for an engineer for field work and systems testing in South Africa, support for a postdoc for
digital correlator development, and partial support for a senior scientist (Carilli, who is supported 
at 20\% time by the Cavendish).

\subsection*{\it University of California, Berkeley}

The UC Berkeley Radio Astronomy Laboratory (RAL) is located adjacent to the UC
Berkeley Astronomy Department. It provides laboratory space and access to
digital and radio-frequency test equipment necessary for the detailed
characterization and performance testing of components of the proposed correlator
development work. Such items include power supplies, signal generators, network
analyzers, oscilloscopes, noise generators, filters, attenuators, amplifiers,
and other miscellaneous electronic equipment and well as machine tools. RAL also hosts the programming
environment targeting the Field-Programmable Gate Array processors on which the
proposed correlator work is based. This facility is available to members of the
UC Berkeley Astronomy Department.

The Berkeley Wireless Research Center is located adjacent to the UC Berkeley
campus. It provides access to additional high-end digital test equipment and
software for digital signal processing.  It is available to members of the
Collaboration for Astronomy Signal Processing and Electronics Research (CASPER).

For the duration of this grant, Berkeley will provide partial support for engineering and management (DeBoer), 
partial support for engineering (both MacMahon and Dexter),
as well as partial summary salary support for (Parsons).

Berkeley will provide a laboratory correlator system for development and
testing of correlators; delay spectrum power spectrum pipeline software; the
94-node Henyey compute cluster for local data processing; Matlab; video
conferencing software to further the collaboration; and a full-size HERA test
antenna.

\subsection*{\it University of Pennsylvania}

The central computing cluster and data archive for PAPER
is maintained at UPenn.  This currently consists of 22 nodes connected by Gb
ethernet. These nodes consist of 16 Dell PowerEdge 1950s (dual 4-core Intel Xeon L5420 \@ 2.50GHz)
and 6 Dell PowerEdge R410 (32 GB RAM; dual 6-core Intel Xeon E5649 \@ 2.53GHz),
for a total of 200 cores and 448 GB RAM.  Fast working data storage is provided
by a network storage system (NSS) based around two Dell MD1200s, with a total
RAID storage capacity of 140 TB.  The storage has a dedicated head node and internal
10Gbe, with 1 Gbe connections to each compute node. Backup of the NSS is done
on to several Silicon Mechanics Storform D59J.v2 machines, each of which is
about 55 TB effective in RAID6.  We currently have three backing up the 140 TB
above. The system was purchased through a combination of the PI's startup
funds, internal funding from UPenn's University Research Foundation, a major
contribution from the Mt.~Cuba Astronomical Foundation, and contributions from
Professors M. Sako and G. Bernstein, who share the cluster for processing of
DES and SDSS data.  Extra space is available to house the additional data
analysis and storage hardware described in this proposal to support the HERA
array.


\subsection*{\it University of Washington}

The University of Washington is providing software developed as a part of the
MWA, including Monitor \& Control software, Fast Holographic Deconvolution
software, and direct power spectrum analysis software.  Morales has led the development of the MWA Monitor \& Control software, and code 
will form the base for developing the associated system for HERA.
The Fast Holographic Deconvolution
software developed by Morales' group
is invaluable for reaching the scientific goals of
HERA, including the survey imaging work and for
obtaining the foreground isolation necessary for the ultimate power-spectrum
analysis and imaging of the EoR with the full HERA instrument.
FHD is currently being used to reduce PAPER imaging data and all MWA Epoch of
Reionization data.
Additionally, UW provides a RF
testing laboratory (vector network analyzers, spectrum analyzers, mixed mode
oscilloscopes, precision system temperature measurements, etc.) and access to
an excellent machine shop.

\end{document}
