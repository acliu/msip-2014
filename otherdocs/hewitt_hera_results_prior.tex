\documentclass[12pt,preprint]{aastex}
\pagestyle{empty}
\begin{document}

Prof. Jacqueline Hewitt (MIT) is PI of one related grant "MRI:  Acquisition of
an Archive for the Murchison Widefield Array (\#0821321, 08/08-7/14) and was a
co-investigator on the now-completed grant that funded the US contribution to the construction
of the MWA "Mileura Widefield Array Science and Technology Demonstrator" 
(\#0457585, PI A. Whitney).  {\bf Intellectual merits:} carried out forecasting analyses that
significantly influenced the design of the MWA; developed and implemented the MWA
receiver's monitor and control; developed real-time and off-line tools for verification of MWA data; 
carried out prototype testing with the 32-element (32T) prototype array; developed a 
CASA-based 32T imaging pipeline leading to first mapping (Williams) and power spectrum
(Dillon et al.) results;  built and implemented US MWA archive, now at 400 TB and planned
to grow to 1 PB when completed; currently receiving and carrying out quality
checks on incoming MWA EoR data. {\bf Broader Impacts:}  The MIT group's MWA participation
has enabled hands-on training of six graduate students (including one woman and
one underrepresented minority student) in radio astronomy,
interferometry techniques, and instrumentation development in the field. 
The MIT/MWA archive will be a resource for the scientific community and the
public when the data proprietary period is complete.
\end{document}