\documentclass[preprint]{aastex} 

\usepackage[top=1in, bottom=1in, left=1in, right=1in]{geometry}
\usepackage{amsmath}
\usepackage{graphicx}
\usepackage{mdwlist}
\usepackage{natbib}
\usepackage{natbibspacing}
\usepackage{caption}
\usepackage{subcaption}
\usepackage{enumitem}
\setlength{\bibspacing}{0pt}
\setlength{\parskip}{0pt}
\setlength{\parsep}{0pt}
\setlength{\headsep}{0pt}  
\setlength{\topskip}{0pt}
\setlength{\topmargin}{0pt}
\setlength{\topsep}{0pt}
\setlength{\partopsep}{0pt}
\setlength{\footnotesep}{8pt}
\pagestyle{empty}
\citestyle{aa}

\newcommand{\simgt}{\stackrel{>}{_{\sim}}}
\def\kperp{k_{\bot}}
\def\kpar{k_{\|}}
\def\k{{\bf k}}
\def\sky{{\theta}}
\def\HI{{H{\small I }}}
\def\HII{{H{\small II }}}
\def\xHI{{x_{\rm\HI}}}

%\usepackage{subfig}
%\usepackage[countmax]{subfloat}

\begin{document}
\title{Project Management Plan}

%A Project Management Plan should be submitted as a supplementary document, and may be up to 
%15 pages in length, although many programs will not need this much space. (Note that the solicitation 
%stated that the management plan should be provided in the Project Description; here we move it to its 
%own supplementary document so that it may be explicitly evaluated by reviewers.)  This section must 
%present a clear and thorough discussion of the project management structure and techniques that will 
%be applied. It may include, for example, a construction plan and schedule, a collaboration management 
%plan, a plan for managing telescope access, or any other pertinent management information. The 
%management plan should identify risks and describe their planned mitigation. If your proposed project 
%would have contributions from sources other than NSF, this section should clearly state both the total 
%cost, and the amount being requested from NSF.  

\section{Collaboration and Governance}
HERA builds upon the organization, tools and collaborations developed within the now merged 
PAPER and MWA-US groups.
In addition to the proposing collaborators in the {\em List of Partner Institutions},
HERA includes key partners in South Africa and the UK and is finalizing a collaboration with ASIAA.
\begin{table}[h]
\begin{tabular}{| p{.35\textwidth} | p{.6\textwidth} |}\hline
\textbf{Institution (location)} & \textbf{Role} \\ \hline
SKA-SA (Cape Town, SA) & Partner in site development, logistics, support, science \\ \hline
University KwaZulu Natal (Durban, SA) & Partner in science and support \\ \hline
Cavendish Laboratory (Cambridge, UK) & Partner in development etc \\ \hline
Academia Sinica Institute for Astronomy and Astrophysics (Taipei, Taiwan) & Science, development \\ \hline
\end{tabular}
\label{tab:otherpartners}
\end{table}

\begin{figure}[h]
\centering
\includegraphics[width=\textwidth]{plots/org.png}
\label{fig:org}
\caption{Org chart showing the collaboration, governance and management.}
\end{figure}

\subsection{Executive Board}
Figure \ref{fig:org} shows the governance and management of the proposed work. 
The Executive Board serves as the governing board and has authority over the science 
Requirements\footnote{{\em Requirements} are the science and operational level requirements as
specified and managed by the Board.  {\em Specifications} are the system requirements
specified by the System Owners.}, the Scope of the science goals and Milestones. It consists 
of one voting representative from each of the funded partners, which typically will be the 
senior scientific partner at the institution. The Board is chaired by the Project Director. Though 
generally operating on concensus, if matters come to a vote a ``T-2''\footnote{T-2 means that 2
or more must vote against to defeat a motion.} supermajority will be required. The
Executive Board will meet at least quarterly, with at least one face-to-face meeting
per year. The Board is responsible for producing the NSF annual report.  The Board has the
authority to name and retract Collaborator status to partners.  An up-to-date list of Collaborators
will be maintained on reionization.org.

\subsection{Science Panel}
A Science Panel, chaired by the Project Scientist or designate, will meet at least
quarterly to provide advice on the science (scope, progress, support, ). The
composition is determined by the Executive Board, and membership is open to anybody
the Board deems advisable and is willing to serve. Under the auspices of the Science
Panel, one face-to-face workshop will be held per year, rotating its location among
the partners.

\subsection{Collaboration Working Group}
The Collaboration Working Group comprises individuals named by the partner
institutions and will meet approximately weekly to review the project. It is chaired
by the Project Director or designate and will discuss any and all relevant aspects of
the project. It represents the primary method of frequent communication between the
partners.

Collaborators are expected to actively participate in HERA, as evidenced by participating
regularly in the weekly Collaboration Working Group meeting and e-mail discussion
threads.  Collaborators should actively assist in publications, in terms of submitting
appropriate material and/or reviewing/editing draft publications.  Collaborators should
seek out and attend appropriate conferences and present HERA/EOR papers.  This will
be coordinated amongst the group to either strategically provide coverage or to encourage 
a large number of team members to attend specific conferences.

Each Collaborator should have a defined project role as fits their interests and organizations,
actively pursue those activities, and communicate these activities to the group.  Collaborators
are encouraged to participate in all aspects of the project, subject to the Publication Policy, 
discussed in Section \ref{sec:publication} below.  In short, the activities should be transparent
to the team.

****ANY STUDENT CARVE-OUTS ETC?

\subsection{Communication}
The core standing meetings for the collaboration have been discussed above, and
collaboration weekly meeting has been underway for some time. This generally is
conducted by the local groups meeting face-to-face and linked together by a
conferencing system. Berkeley provides full access to the Blue Jeans
network-conferencing system for hosting meetings, including video, simple and
effective desktop sharing, and multiple access methods (computer, phone, networked
video system). Other partners have access to Webex,which is also used. These tools
are routinely and effectively used by the partners already and are quite effective.
Other side meetings also occur using these tools, including regular telecons
specifically with South Africa The collaboration makes good use of side-meetings at
professional conferences at which many of the team may be attending.
In addition to meetings, the group obviously makes copious use of e-mail, both direct 
and using a group listserver.  Other tools, such as wikis and web-sites are discussed below.

\subsection{Publication}
\label{sec:publication}
Any paper reporting the detection of a 21cm EoR signature, or reporting a significant
improvement in upper limits, will have the authorship of the full Collaboration, but
excluding those Collaborators who choose not to be on the author list. The paper
acknowledgment section is an appropriate mechanism for identifying individual
contributions of authors.

Other publications that either refer to a technique, related sensitivity analysis, system
description, use of data for other science, etc are also encouraged.  The author list
may vary, but broad collaboration acknowledgement is encouraged, and the list should
be discussed prior to submission.

\section{Management}
Figure \ref{fig:org} also shows the management structure.   A Project Manager has
overall responsibility for planning and tracking to achieve the agreed upon
milestones and budget. The Project Manager will brief the Executive Board at each
meeting to summarize the state of the project.  David DeBoer from Berkeley will serve
as the Project Manager.

The Site Liaison is a South African-based person responsible for coordinating activities
between the Collaboration and the broader SKA-SA observatory group.  This link has 
been key in effectively pursuing the PAPER activities in South Africa.  William Walbrugh

The Data Manager has responsibility for ensuring data quality as archived at the KAPB and 
Penn.  James Aguirre from Penn will serve as the Data Manager.

The Analysis Manager has responsibility for ensuring that the analysis pipelines are super terrific.
Miguel Morales from Washington will serve as the Analysis Manager.

The Project Engineer has responsibilty for ensuring that Components and Systems meet their overall
Specifications and that Interfaces are properly addressed.  Bob Goeke from MIT will serve as the
Project Engineer.

\subsection{Project Structure}
The project comprises {\em Components}, which are the physical deliverables, and {\em Systems},
which are logical groupings to achieve a function. Systems may be hardware, software
or both. Systems need not be disjoint collections of Components. Each System has an
{\em Owner}, who is responsible for the System meeting its specifications and satisfying
its interface requirements.

Components will likely span multiple Systems. The Project
Manager serves as the Owner of the South Africa-based hardware Components, the data
manager is the Owner of the US-based hardware Components and the analysis manager is 
the Owner of the analysis software.  The Component Owners work with the System Owners 
to meet the Components functional specifications.  In practice there is a great deal of overlap
between the Components and Systems, making this arrangement practical but it also allows
a mechanism to ensure more ``eyes'' on the overall system.

\subsection{Management Tools}
The architecture, requirements/specifications, interfaces, budget, milestones, work breakdown structure
(WBS) and system documenation are in a common shared repository, along with collaboration
documentation.  This shared repository is referred to as the {\tt Project Book}.  Architecture, requirements/ 
specifications, interface and milestone information reside in a \LaTeX-based database with full traceability
and python and shell scripts for flow-down throughout the documentation.  Budget and WBS are in Microsoft
Excel and Microsoft Project with python script readers for flow-down throughout the documentation.

An executive script ensures that all scripts are executed and up-to-date formatted \LaTeX pdf documentation
files are produced.  These include auto-generated architecture, requirement/ specification, 
interface control, and milestone documentation, budget and WBS summary documents, as well as
flow-down files into the architecture for traceability. 
In addition to the auto-generated reports, System and  Component Owners write the narrative 
documentation referencing the database items as variables ({\em e.g.} requirements, interfaces, etc).
Additional scripts may be written as needed to access the system for customized reports.

Frequent telecons with modern conferencing systems are absolutely essential collaboration tools, 
as discussed above, and liberal use of these systems will continued to be applied.

The Project Book remains the definitive record, but use of a wiki for posting material and hosting
discussions etc is an important component.  The project has full access to wiki tools, and has been
using collaborative wikis in the ongoing collaboration.  The group has a public web-site at http://reionization.org, 
in addition to PAPER and MWA project web-site.

Additional common repositories host the software developed by the project.  This allows access to all members,
as well as full revision control.  Git/Github is the primary repository tool.  Python is the primary coding/scripting
tool.

Financial tracking is done via the Berkeley Financial System (BFS), supplemented with Excel spreadsheets.

Table \ref{tab:softwareTools} lists the primary software tools (management and technical).  Schematics:  visio, technical
\begin{table}[h]
\centering
\label{tab:softwareTools}
\caption{Project Software Tools}
\begin{tabular}{| p{1in} | p{5in} |}\hline
\textbf{Category} & \textbf{Tools (primary is in bold font)} \\ \hline
Documentation & \textbf{\LaTeX}, Word, {\tt ProjectBook}, pdf \\ \hline
File sharing & \textbf{Github}, googledocs \\ \hline
Scheduling & \textbf{MS Project}, ganttProject \\ \hline
Finance & \textbf{BFS}, Excel \\ \hline
Scripting & \textbf{Python}, shell \\ \hline
Technical drawings & autocad (Draftsight), Autodesk Inventor \\ \hline
Diagrams & \textbf{Visio}, Omnigraffle \\ \hline
\end{tabular}
\end{table}


\subsection{Change Control}
\label{sec:changecontrol}
Change requests are initiated by an e-mail to the Project Manager indicating the proposed change
and reasons why it is necessary.  Changes may be of the Architecture, Scope, Requirements/ Specifications
Milestones and/or Interface as documented in the Project Book.  The Project Manager will investigate the 
impact of the proposed change and make a recommendation which may be provided to different bodies, 
depending on the scope of the change.  The Project Manager may initiate a change request by appropriately 
formulating and sending on a recommendation.

As mentioned above, the Executive Board controls Requirements, Scope and Milestone change requests
and they receive recommendations regarding these items.  If accepted, the Board will instruct the Project Manager to 
propagate the change within the management system and communicate this change to the partners.

Other change request recommendations will be provided to the impacted Owners of Systems and
Components and accepted on a consensus basis of the impacted Owners and Project Manager.  
In the event that consensus is not reached, the Project Director remains the final arbiter.  The Project 
Director may overrule consensus by approval of the Executive Board.  If accepted, the Project Manager 
will institute the change and communicate appropriately throughout the project.

\section{Array Construction}
\label{sec:construction}
The primary construction location is in the South African Karoo Astronomy Reserve, where PAPER has been deployed
and operated since 2009.  The South African SKA organization (``SKA-SA'') is a key partner and has a great deal of
infrastructure and support on-site and for the site and a close working relationship exists between the partners.

The project is phased and designed to maximize off-site sub-assemblies, which are then shipped to site for subsequent
installation.  The on-site installation has an overall On-Site Manager (which rotates among a small number of expert
staff) supervising two 6-8 member crews, each with a crew chief.  Primary sub-assemblies are the hub casts, pole/post end-caps 
(which hold the surface support spars and rim pieces and contain locational markers), PVC support sub-assemblies 
for the surface, feed, feed-backplane, central cone, node enclosure.XXX

\begin{table}[tbh]
\centering
\label{tab:subassycontracts}
\caption{Primary sub-assembly contracts}
\begin{tabular}{| p{1.4in} | p{1.4in} | p{1.4in} | p{1.4in} |}\hline
\textbf{Element} & \textbf{Analog} & \textbf{Digital} & \textbf{Node} \\ \hline
%Element
hub casts, pole/post endcaps, PVC surface support, feed, feed backplane, central cone &
%Analog
LNA/balun modules, post-amplifier modules, RF cables &
%Digital
SNAP boards &
%Node
Enclosure, power handling \\ \hline
\end{tabular}
\end{table}

The ordered list of installation is given below.  A number in square brackets denotes a separate contract (or sub-contract),
while a single asterisk denotes work that may be done in parallel by the two teams.  An additional team (or teams) may be 
added to pace the work, primarily in the 127-to-352 phase (when there will also be more experiences members of the team
who may step up as crew chiefs).  Double asterisks are activities done with US experts supported by local staff.  The first
installations will also involve US-based experts to train the on-site staff, particularly the crew chiefs.
\begin{enumerate}[itemsep=-3pt]
\item Sub-assembly contracts [1]
\item Grounds preparation and power/fiber reticulation. [2]
\item Pole installation [3]
\item Hub installation using centering and hub jigs  [*]
\item Rim/post installation using leveling jig and Total Station [*]
\item Surface support installation [*]
\item Mesh installation [*]
\item Node enclosure [*]
\item Feed installation [**]
\item Node electronics [**]
\item KAPB electronics [**]
\item Commissioning [**]
\end{enumerate}

The overall schedule and milestones are summarized in section \ref{sec:schedule}


\section{Analysis Management}
\label{sec:analysis}

\section{Budget Summary}
\label{sec:budget}
\begin{table}[t]
\centering
\caption{System Cost Summary}
\label{tab:budgetsummary}
\begin{tabular}{| p{2in} | p{2in} | p{2in} | }\\ \hline
\noindent
\textbf{Element:}  \$2,064,411
\begin{itemize}[parsep=-2pt, itemsep=-3pt]
\item Misc:   \$170,474
\item Hub:   \$26,400
\item Support:   \$576,426
\item Surface:   \$729,696
\item Labor:   \$481,416
\item Shipping:   \$80,000
\end{itemize}
 &
 \noindent
\textbf{Frontend:}  \$498,487
\begin{itemize}[parsep=-2pt, itemsep=-3pt]
\item Misc:   \$5,000
\item Feed:   \$120,384
\item LNA:   \$293,959
\item RFcable:   \$64,064
\item Power:   \$7,040
\item Labor:   \$7,040
\item Shipping:   \$1,000
\end{itemize}
 &
 \noindent
\textbf{Node:}  \$1,254,981
\begin{itemize}[parsep=-2pt, itemsep=-3pt]
\item Misc:   \$26,000
\item Enclosure:   \$123,500
\item Receiver:   \$601,061
\item DAQ:   \$206,144
\item Control:   \$52,000
\item Fiber:   \$78,000
\item Power:   \$45,500
\item RFcable:   \$83,776
\item Labor:   \$13,000
\item Shipping:   \$26,000
\end{itemize}
\\ \hline
\noindent
\textbf{Container:}  \$834,425
\begin{itemize}[parsep=-2pt, itemsep=-3pt]
\item Misc:   \$5,000
\item Fibre:   \$129,787
\item Power:   \$425,300
\item Timing:   \$15,000
\item Site:   \$228,838
\item Control:   \$500
\item Shipping:   \$30,000
\end{itemize}
 &
 \noindent
\textbf{KAPB:}  \$571,999
\begin{itemize}[parsep=-2pt, itemsep=-3pt]
\item Misc:   \$5,000
\item Datahandling:   \$232,874
\item Processor:   \$228,200
\item Control:   \$3,000
\item Fiber:   \$47,200
\item Power:   \$30,225
\item Labor:   \$15,500
\item Shipping:   \$10,000
\end{itemize}
 &
 \noindent
\textbf{Computing:}  \$457,393
\begin{itemize}[parsep=-2pt, itemsep=-3pt]
\item Misc:   \$10,000
\item Cluster:   \$187,166
\item Storage:   \$255,227
\item Server:   \$5,000
\end{itemize}
\\ \hline
\end{tabular}
\end{table}



\begin{table}[h]
\centering
\label{tab:expenses}
\caption{Expenses summary}
\begin{tabular}{| p{0.5in} | p{.6in} |  p{.6in} |  p{.6in} |  p{.6in} |  p{.6in} |  p{.6in} |  p{.6in} |  p{.6in} | }\hline
  k\$   & \textbf{Berkel} & \textbf{Penn} & \textbf{MIT} & \textbf{UW} & \textbf{ASU} & \textbf{UCLA} & \textbf{NRAO} & \textbf{TOTAL}\\\hline
\textbf{Salary}&       2,882  &         470  &         767  &         282  &         514  &         241  &         168  &       5,323  \\\hline
\textbf{Equipm}&       5,253  &         459  &          24  &          10  &          28  &           0  &          67  &       5,840  \\\hline
\textbf{Other}&         711  &         116  &         181  &         126  &         135  &           1  &          10  &       1,280  \\\hline
\textbf{Indire}&       2,032  &         316  &         441  &         106  &         353  &         104  &          17  &       3,370  \\\hline
\textbf{TOTAL}&      10,878  &       1,362  &       1,413  &         524  &       1,030  &         346  &         261  &      15,813  \\\hline
\end{tabular}
\end{table}



\section{Schedule Summary}
\label{sec:schedule}

\section{Risk}
\label{sec:risk}
\end{document}
