\documentclass[preprint]{aastex} 

\usepackage[top=1in, bottom=1in, left=1in, right=1in]{geometry}
\usepackage{amsmath}
\usepackage{graphicx}
\usepackage{mdwlist}
\usepackage{natbib}
\usepackage{natbibspacing}
\setlength{\bibspacing}{0pt}
\setlength{\parskip}{0pt}
\setlength{\parsep}{0pt}
\setlength{\headsep}{0pt}  
\setlength{\topskip}{0pt}
\setlength{\topmargin}{0pt}
\setlength{\topsep}{0pt}
\setlength{\partopsep}{0pt}
\setlength{\footnotesep}{8pt}
\pagestyle{empty}
\citestyle{aa}

\newcommand{\simgt}{\stackrel{>}{_{\sim}}}
\def\kperp{k_{\bot}}
\def\kpar{k_{\|}}
\def\k{{\bf k}}
\def\sky{{\theta}}
\def\HI{{H{\small I }}}
\def\HII{{H{\small II }}}
\def\xHI{{x_{\rm\HI}}}

\def\EnumCompress{\parskip-4pt}

%\usepackage{subfig}
%\usepackage[countmax]{subfloat}

\begin{document}
\title{Postdoc Mentoring Plan}

\section{University of California Berkeley}
Post-docs are key members of the active EOR research group at Berkeley and they have worked and will 
continue to work closely with faculty and staff as part of a team.  Post-docs are integral part of the HERA
collaboration and have been leaders in defining the shape of the science and instrument.  This important
relationship is the key ingredient to a successful post-doctoral tenure.

Post-docs live and work within the complete Astronomy Department and are encouraged and 
expected to attend and present at the may colloquia, lunch and seminars that occur within the 
Department.

Post-docs are also an integral part of the greater University community here at Berkeley

\section{MIT}
The research groups of Professors Tegmark and Hewitt work closely together on 21cm
cosmology, with frequent joint meetings between the two groups, and cross
fertilization with students doing projects associated with both the Murchison
Widefield Array and the Omniscope. As advisors, we are committed to guiding the
postdoc in her/his research program, to advising her/him on how best to excel in and
benefit from the MIT environment, and to preparing her/him for a career in
astrophysics or a related technical and/or scientific field. We have frequent contact
with our postdocs and students through weekly group meetings, one-on-on meetings,
informal collegial interactions with the broader astrophysics group at the MIT Kavli
Institute (MKI), and various social events.

The postdoc will also be expected to participate in the broader astrophysics
community at MIT, for example participating in (and perhaps at times helping to lead)
the MKI Astrophysics Journal Club, participating in the annual MKI Postdoctoral
Research Symposium (by presenting her/his research), and attending weekly
Astrophysics colloquia. S/he will also participate in periodic (2-3/year) MKI postdoc
lunches with faculty, and in biweekly lunches that bring together the postdocs to
discuss their research and other topics among themselves. These lunches provide
networking opportunities and may include guest speakers on topics (such as “Getting
Your First Faculty Position”) that are of interest to postdocs.

Finally, s/he will be part of the larger postdoc community at MIT. The Office of the
Provost at MIT maintains an informational website
(http://web.mit.edu/mitpostdocs/index.html) for postdocs. They also sponsor periodic
educational, training and social events and maintain a postdoc listserv for
disseminating information to postdocs throughout the Institute.

\section{University of Washington}
Close mentoring of postdoctoral scholars is an integral component of our research
group. Our goal is to provide scholars with the support and guidance to grow
professionally and be ready for positions in either academia or industry, and to
tailor their experiences to their individual professional goals. For every scholar, I
have a meeting when they arrive to discuss their professional goals and to outline
the projects and experiences that will get them there. This is codified as a set of
annual goals, which is revisited once a year as part of an annual evaluation.

The research environment of our group is quite interactive, with frequent interactions
(typically once a day). This is supplemented with twice weekly video group meetings
with the full research team. In addition to standard opportunities for growth
including leading specific projects, first authorship on collaboration papers,
conference presentations, and proposal/fellowship applications, we make a particular
point of helping scholars develop the interpersonal skills and connections needed to
succeed. All scholars have the opportunity to lead small teams on independent
projects where the scholar oversees one or more graduate or undergraduate students to
develop the scholar’s mentorship and management skills. Additionally the scholars
work within the HERA collaboration and have the opportunity to work with many of the
leading researchers in radio cosmology. 

In addition, our group is quite diverse and creating a welcoming and supportive
atmosphere for all the members is very important to us. By modelling how an effective
and diverse group should interact—high expectations in a supportive and collaborative
atmosphere—I help my postdoctoral scholars learn how to effectively collaborate and
lead groups of their own.


\end{document}
