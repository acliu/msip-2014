%Feb 5, 2014
%Draft outline: MSIP
%Chris Carilli
\documentclass[preprint]{aastex}

\usepackage[top=1in, bottom=1in, left=1in, right=1in]{geometry}
\usepackage{amsmath}
\usepackage{graphicx}
\usepackage{mdwlist}
\usepackage{natbib}
\usepackage{natbibspacing}
\usepackage{caption}
\usepackage{subcaption}
\setlength{\bibspacing}{0pt}
\setlength{\parskip}{0pt}
\setlength{\parsep}{0pt}
\setlength{\headsep}{0pt}
\setlength{\topskip}{0pt}
\setlength{\topmargin}{0pt}
\setlength{\topsep}{0pt}
\setlength{\partopsep}{0pt}
\setlength{\footnotesep}{8pt}
\pagestyle{empty}
\citestyle{aa}
\usepackage{enumitem}
%\usepackage[font=small]{caption}


\newcommand{\compress}{\vspace{-0.12in}}


\def\kperp{k_{\bot}}
\def\kpar{k_{\|}}
\def\nwnh{{\sl NWNH}}

\newcommand{\simgt}{\stackrel{>}{_{\sim}}}
\def\kperp{k_{\bot}}
\def\kpar{k_{\|}}
\def\k{{\bf k}}
\def\sky{{\theta}}
\def\HI{{H{\small I }}}
\def\HII{{H{\small II }}}
\def\xHI{{x_{\rm\HI}}}

%project description 20 pages total

\begin{document}

\title{Hydrogen Epoch of Reionization Array: Characterizing Cosmic Dawn}

\section{Overview} % 1 page ~ project summary? 
% Parsons, Carilli

% A statement of which of the four categories of MSIP is most appropriate
%for this proposal as the first sentence (see section II. Program Description).

%A. We propose HERA: next step in reionization roadmap
%B. Fulfill NWNH high-priority goals
%C. New understanding/techniques => faster, better, cheaper
%D. Brief summary of timeline: science along the way, major results before end-decade

{\it For the Mid-Scale Science Projects category of the Mid-Scale
Innovations Program}

The Hydrogen Epoch of Reionization Arrays (HERA) roadmap is a staged
program that uses the unique properties of the 21-cm line from neutral
hydrogen to probe the Epoch of Reionization and the preceding Dark
Ages.  During these epochs, roughly 0.3--1~Gyr after the Big Bang, the
first stars and black holes warm and reionize the neutral
intergalactic medium (IGM) that pervades the Universe following cosmic
recombination. Direct observation of the large scale structure of
reionization and its evolution with time, via the \HI 21-cm line, will
have a profound impact on our understanding of the birth of the first
galaxies and black holes, their influence on the intergalactic medium
(IGM), and cosmology.  CMB comparison statement? 

HERA was ranked the ``{\it top priority in the Radio, Millimeter, and
Sub-millimeter category of recommended new facilities for mid-scale
funding}" as part of the {\it New Worlds, New Horizons of Astronomy
and Astrophysics} decadal survey (\citealt{astro2010}; hereafter
NWNH).  The HERA roadmap initially envisioned a series of radio
interferometers constructed throughout the decade, starting with the
existing Donald C. Backer Precision Array to Probe the Epoch of
Reionization (PAPER) and the Murchison Widefield Array (MWA)
instruments aimed at characterizing foregrounds and laying the
groundwork for a statistical detection of the HI 21cm signal through
the power spectrum.  A second-generation HERA instrument would measure
the evolution of the power spectrum in detail and reveal how early
structure in the Universe formed. A third-generation instrument would
image the typical structures during reionization.

While receiving only a fraction of the funding for HERA phase I
recommended in NWNH, the MWA and PAPER projects have made
major strides in understanding the techniques required to disentangle
the reionization signal from the strong radio continuum foreground
confusion. Based on this new understanding of array response to the
celestial signal, we are ready to build the next generation of HERA in
stages of 127 and 331 elements, observing in the 50--225-MHz band.
These stages increase the sensitivity by two orders of
magnitude over to the current arrays, through the use of an
optimal configuration of 14m diameter receptor elements, and a
tailored analysis technique that mitigates foreground contamination.
The proposed experiment delivers HERA-II science at a cost
substantially below originally envision in NWNH:

% might remove 1st phrase in 1st sentence in para above

\vspace{-4pt}
\begin{itemize}\setlength{\parskip}{0pt}\itemsep0pt

\item HERA~127 will measure the rise and fall of the EoR power
spectrum, constraining the timing and duration of reionization.

\item HERA~331 will measure EoR fluctuations over a variety of spatial
scales to determine the features and distribution of the first objects
that dominate cosmic reionization. HERA~331 will also extend precision
power-spectrum observations into the Dark Ages, and directly image the
largest scale structures in the IGM during reionization.

\end{itemize}
\vspace{-4pt}

The proposed program produces a sequence of dedicated experiments
optimized to fulfill the NWNH goal of characterizing the evolution of
the HI 21cm power spectrum during cosmic reionization. In
its final stages, HERA may be capable of imaging the neutral
IGM --- a task previously only considered for third-generation
instruments. This proposal includes funding for both the
telescope development and construction, as well as for the key
scientific data analyses for each stage of the project.

\section{Science} total = 8 pages

\subsection{Introduction}    % 1 page + 1 fig = simulated Tb cube + PS evolution
% Furlanetto, Carilli

%i.physical concepts: reionization and dark ages
%ii. Current knowledge: various constrai	nts, 1st galaxy studies...
%iii. Important role of 21cm studies highlighted in NWNH
%a. Typical ideal sim results: T_B vs. z 'cube' and corresponding power spectrum evolution (Fig)
%b. introduce some of the important parameters/processes explored: when? how? bubble scale? sources? inside out? [much of this could be left for below?)
%c. reemphasize that current knowledge is nil, yet demand is high
%d. emphasize unique (only?) probe of dark ages

The {\it cosmic dawn}, or the period beginning with the the birth of the first stars and culminating with the full
ionization of the intergalactic medium (IGM) some 500 Myrs later, represents one of the last unexplored phases in 
the history of structure formation. During this period, a wealth of astrophysical and cosmological phenomena are at 
work. The characteristics of the IGM depend on (for example) the cosmic density field, the formation sites of the 
first luminous sources (e.g., their typical masses and clustering), their constituents (e.g., exotic Population III 
stars, more normal stars, stellar remnants, or supermassive black holes) their ultraviolet luminosities (which affect 
the IGM's ionization state), the efficiency and abundance of X-ray sources (which affect the IGM temperature), and 
even exotic effects like the relative velocity of baryons and dark matter.  Exploring these early structures and their 
effects on each other and their environments was one of the top three ``{\it priority science objectives chosen by 
the [NWNH] survey committee for the decade 2012-2021.}"

%SF: This next paragraph worries me a bit because most of these probes have one or several competing groups working 
%on them, and if we get a referee from one of the groups we don't reference they might get pissy...right now 
%I've tried to be fair but terse here
Observations are now just beginning to penetrate into this early epoch, primarily at its tail end.  Recent 
measurements from the {\it Hubble Space Telescope} have pinned down the bright end of the galaxy luminosity function 
at $z \la 8$ \citep{bouwens_et_al2010, schenker_et_al2013} and begun to detect a few sources at even greater 
distances \citep{ellis_et_al2013, oesch_et_al2013}.  Meanwhile, a number of indirect probes have placed constraints -- 
albeit model-dependent ones -- on cosmic reionization, as summarized in the left-hand panel of 
Figure~\ref{fig:x_i_Xray}. These include observations of resonant scattering of Ly$\alpha$ by the neutral IGM toward
distant quasars (the 'Gunn-Peterson' effect) \citep{fan_et_al2006}, the demographics of Ly$\alpha$ emitting 
galaxies \citep{schenker_et_al2012, treu_et_al2013}, and Ly$\alpha$ absorption near the most distant quasar 
\citep{bolton_et_al2011}, as well as measurements of CMB temperature fluctuations \citep{planck_et_al2013}, 
polarization anisotropies \citep{page_et_al2007}, and secondary temperature fluctuations generated by the 
kinetic Sunyaev-Zel'dovich effect \citep{zahn_et_al2012_trunc, mesinger_et_al2012}.  Reionization models can be 
built from galaxy observations.  Recent examples suggest that the IGM may have been substantially neutral even at 
$z \sim 7$--8 (as high as $F_{HI} \sim 0.5$), with a tail of finite ionization ($\sim 0.1$) extending to high 
redshift ($z \sim 15$), driven by very early galaxy formation \citep{robertson_2013}.  The extrapolation to 
$z>8$ is particularly uncertain, as only a handful of candidate galaxies have been found during that earlier 
era (c.f. \citealt{oesch_et_al2013}).  Unfortunately, these important methods have limited reach: strong assumptions 
are required about the stellar populations and escape fraction of UV photons from the galaxies, while astrophysical 
constraints on reionization are highly model-dependent and the CMB provides only an integral measure of the process. 
Moreover, many of these indirect observations are in tension with one another, underscoring both the difficulty in 
their interpretation and the complexity of the reionization process.

The 21-cm ``spin-flip" transition of  neutral hydrogen has been recognized as potentially the most powerful probe 
of cosmic reionization and  the dark ages \citep{morales_wyithe2010, furlanetto_et_al2006}, as emphasized in NWNH: 
``{\it 
%SF: Is this quote from the RMS panel?  If so, does it appear in the final report?
The panel concluded that  to explore the discovery area of the epoch of reionization, it is most important to 
develop new capabilities to observe redshifted 21-cm \HI emission, building on the legacy of current projects and 
increasing sensitivity and spatial resolution to characterize the topology of the gas at reionization.}"  Because 
the Universe is almost entirely neutral before reionization, the HI 21-cm line provides a direct method to image 
the evolution of the primordial IGM, opening a unique window into the complex astrophysical interplay between the 
first luminous structures and their surroundings. Moreover, because of the cosmological redshift, we can associate 
the signal at each observed frequency with a particular emission time (or distance) and reconstruct a complete 
three-dimensional picture of the time evolution of large scale structure during this 'last frontier' in 
observational cosmology. 
%SF: Personally I feel like the following is an exaggeration that could rub some people the wrong way!
The direct observation of the primordial IGM via the HI 21-cm line would be an achievement comparable in importance 
to the detection of anisotropies in the CMB.

In the past decades, considerable effort has gone into modeling the complex astrophysics of reionization
(e.g. \citealt{shapiro_giroux1987, haiman_loeb1997, furlanetto_et_al2004, santos_et_al2010}). Figure ?? shows a 
simulation of the expected evolution of the HI 21cm signal during reionization. The HI 21-cm fluctuations initially 
rise above those expected from the cosmic density field due to the growth of ionized bubbles on a characteristic 
scale of a few to 10~arcmin. This scale is set by the clustering of early galaxy formation, as well as by 
propagation effects through the IGM. The signal then declines as the IGM becomes fully ionized.  These theoretical 
models of the reionization process are sophisticated and appear to be well-understood, {\it but they are not 
predictive tools.} Instead, they provide a mapping between the largely unknown galaxy populations and observables 
like the 21-cm transition. As such, the key questions about the cosmic dawn era remain poorly understood.  When 
did reionization occur, and over what timescale?  What objects dominated the radiation field?  How were the 
objects distributed?  What were the most important feedback mechanisms in the transition from the first stars to
first galaxies, and how did they affect these populations?  {\it HERA provides the key measurements that are needed 
to advance our understanding of early galaxy formation and cosmic reionization.}


Figures: F(HI) vs z;  HI Tb cube + PS evolution


\subsection{HI reionization and dark ages science} % 4 pages total examples using hera 331, plus a number of figures

i. F(HI) vs. z: HERA vs. other techniques (Fig) -- emphasize constaints in unexplored territory
% Bowman


\subsubsection{Detecting and characterizing the power spectrum}
\emph{ii. PS sensitivity at fixed z: 127, 331 (Fig)}
% Pober, Dillon
\begin{figure}[t]\centering
\includegraphics[width=\textwidth]{plots/Pspec/eor_pspec_2014.png}
\caption{Power-spectrum sensitivities for three stages of
HERA (solid) relative to a fiducial ionization model (dotted line; $\xHI=0.37$, $z=9.0$).  
Sensitivity curves reflect a staged array size and
a staged improvement in analysis software that expands the range
of modes falling into the EoR Window \label{fig:PspecSensitivity}}
\end{figure}

A season of observing with HERA-127 will yield high-significance constraints on the 21 cm power spectrum across a wide range of k modes and redshifts.  In Figure we show the $z=9$ power spectrum predicted by the publicly available 21cmFAST software \citep{mesinger_et_al2011}, along with $2\sigma$ HERA sensitivities.  Using the conservative delay-spectrum approach employed in Parsons et al. 2014, we find that HERA-127 can achieve a $> 10\sigma$ detection of fiducial power spectra over a broad range of redshifts.  The subsequent observing season with HERA-331 can increase this detection significance to over $25\sigma$ using the same methods.  With detailed foreground modeling, a more sophisticated power spectrum estimator could increase the size of the ``EoR window", the region of Fourier space with minimal foreground contamination. This would allow for an overall detection significance of up to $90\sigma$, along with access to lower $k$ modes and therefore qualitatively different physics.  Such a high sensitivity measurement would also allow one to go beyond constraining parameters, testing rather than assuming the underlying theoretical framework.

\subsubsection{Astrophysical parameters from the power spectrum}
\emph{iii. Various covariance analyses: constraints on different physical processes (Liu/Pober analysis). but
please -- keep this down to a couple of incisive figures and fiducial models. wall paper doesn't sell very well. }
% Liu, Pober, Dillon
The power spectrum measurements with HERA-331 are sensitive enough to place constraints on theoretical models that describe the reionization process.  To first order, the major features of the power spectrum (as simulated by 21cmFAST) can be parameterized by three terms: $\zeta$, the efficiency at which galaxies release ionizing photons into the IGM; $T_{\rm vir}$, the minimum virial temperature of halos that produce ionizing photons (a proxy for the minimum mass of the galaxies that drive reionization); and $R_{\rm mfp}$, the mean free path for ionizing photons traveling through the IGM, which is determined the prevalence of dense Lyman limit systems.  Current observations limit the value of these parameters to within an order-of-magnitude (or worse, in the case of $T_{\rm vir}$).  Figure \ref{fig:ErrorEllipses} shows the constraints on each of these parameters achievable with multi-redshift HERA-331 power spectrum observations.  We expect to constrain these parameters to better than 5\% with a conservative approach to foregrounds , and even better with explicit foreground modeling \citep{pober_et_al2014}.

\begin{figure*}[t]\centering
\includegraphics[width=\textwidth]{plots/Pspec/OPTMIDellipses.pdf}
\caption{Pairwise $2\sigma$ error ellipses for $T_{\rm vir}$, $\zeta$, and $R_\textrm{mfp}$, in each case divided by their fiducial values.  HERA-331 projections using existing foreground avoidance techniques are shown in red, while projections using more advanced foreground modeling techniques are shown in black.  The former represent $\sim 5\%$ constraints, while the latter represent $\sim 1\%$ constraints.\label{fig:ErrorEllipses}}
\end{figure*}


% Jacobs
\subsubsection{Imaging HI}
iv. Imaging large scale structure: simulation (Fig) 
With a nearly completely sampled aperture over 300m across, HERA will have the collecting area of Arecibo but with 500x the survey speed it presents the opportunity for directly imaging reionization.  After 100 hours on a single field (achievable in 200 nights, or just over one season) HERA will reach a surface brightness sensitivity of 50 $\mu$Jy compared with typical brightness temperatures of EoR models which often peak above 400 $\mu$Jy.

In imaging, as in the measurement of the power spectrum, noise and foreground residual are comparable limiting factors. Using the foreground filtration methods developed for power spectrum estimation, we can remove foregrounds exactly as done for the power spectrum estimation --filter the ``wedge'' in power spectrum space-- before imaging.  This ``foreground avoidance'' scheme effectively limits the bandwidth (line of sight modes) over which the residuals can be imaged without signal loss.  In Figure \ref{fig:imaging} we demonstrate the sensitivity of imaging in this mode assuming a conservative 1MHz of effective bandwidth, 12 Mpc along the line of sight.  The brightest structures in the redshift 8 input simulation are detectable at the SNR$>10$ level.

\begin{figure}[t]\centering
\includegraphics[width=\textwidth]{plots/Imaging/HERA_FoV.jpg}
\caption{The HERA stripe.  At 150MHz ($z=8.5$) the HERA field of view is 8\arcdeg.  With a nearly completely sampled aperture over 300m across, HERA will have the collecting area of Arecibo but with 500x the survey speed. Each night it will drift scan 2600 square degrees for a survey volume of 50 $Gpc^3$.  The stripe includes the GOODs south field, one of the best studied regions of sky. \label{fig:HERA_FoV}}
\end{figure}

With a resolution of 60Mpc and survey area of 1 Gpc$^2$ in a single field of view HERA images will probe structure on scales well beyond any deep, high redshift field contemplated. For comparison, note that the entire Great Observatories Origins Deep Survey (GOODS), where 40\% of all objects at $z > 6$ have been found, is only (30Mpc ) 15{\arcmin } across. Using deep HERA images it will be possible to make targeted observations of early galaxies known to be in the center of large scale bubbles, directly observe clouds responsible for Ly$\alpha$ absorption, and correlate large scale intensity surveys.


 
\begin{figure}[t]\centering
\includegraphics[width=\textwidth]{plots/Imaging/HERA_331_z8_SNR_annotated.jpg}
\caption{\small
With sensitivity highly concentrated at the largest scales, HERA is capable of directly imaging HI during reionization.  Shown here is a simulation of EoR emission (McQuinn 2009) as imaged by HERA with noise equivalent to 100 hours of observation and bandwidth equivalent to the moderate foreground model. %XXX are we using the language of pober et al? 
Contours enclose regions with signal to noise above 10.  The regions detected on scales of $\sim$100 Mpc are bracket the size scales probed by deep galaxy surveys (cf. the GOODs-South survey volume where 40\% of all galaxies above redshift 7 have been detected.)  The GOODs field itself is located within the HERA stripe, just 3 degrees outside of the image shown here.
\label{fig:imaging}}
\end{figure}    

\subsubsection{Imaging as a probe of non-gaussianity}
\emph{a) Imaging as a probe of non-Gaussianity and topology of reionization (Fig from Watkinson \& Pritchard)
[Not sure if this belongs here, should discuss: b) Bayesian imaging (Figs from Paul Sutter).  
Perhaps a broader impact on the radio community too?]}
% Morales, Tegmark

\subsubsection{Early IGM heating}
\emph{v. Approaching Dark ages (z=20 to 30): early Xray heating? other (Fig - Liu models)}
% Liu, Dillon, Hewitt
\begin{figure}[t]\centering
\includegraphics{plots/Xray/HERA_II_compare_kp1_whoriz_20pt.pdf} 
\caption{\small 
At low frequencies, HERA opens a window to
pre-reionization physics at the end of the Dark Ages. Plotted are power spectrum amplitudes (at $k =
0.15h$~Mpc$^{-1}$) for various IGM heating models \citep{mesinger_et_al2013},
with predicted HERA sensitivities.
}\label{fig:Xray} \end{figure}

With high thermal noise sensitivity throughout the observing band, HERA represents an opportunity to push the redshift frontier of current-generation instruments, extending observations to the pre-reionization era in an uninterrupted way.  Doing so will allow a measurement of an earlier peak in the $21\,\textrm{cm}$ power spectrum, corresponding to an era of IGM heating from various X-ray sources.  Theoretical expectations span a wide range of possible scenarios for IGM heating, and Figure \ref{fig:Xray} shows several that were examined in \cite{mesinger_et_al2013}, with HERA's sensitivity overlaid.  It is clear that HERA possesses the sensitivity to easily detect and distinguish between these possibilities, placing the first observational constraints on the pre-reionization epoch, including possible bounds on exotic physics such as dark matter annihilation.


vi. 21cm forest: perhaps a paragraph on possibilities of small scale structure, if we can find radio galaxy? (Fig) 
% Carilli, Furlanetto -- Forget it.  Too much explanation needed. 

vii. Cross-correlation science: perhaps montage Fig showing HI(Tb), galaxies, dark matter... at given epoch
\cite{lidz11}
% Aguirre, Tegmark

a. Provide environmental context for ALMA/JWST 1st galaxy studies

b. CMB pol 

c. CO/CII IM: optimistic or wrong timescale?
  
d. note: xcorr further mitigates continuum systematics

e. Anscillary science with PAPER 128: transients, solar 
% de Boer

\subsection{Where we are right now: PAPER, MWA}  % 2 pages

i. describe arrays, state design driven by new understandings as delineated below (Fig)
% Parsons, Bowman

ii. Using new techniqes, current best limits  (Fig)
% Parsons, Morales

iii. what will happen in next 2 years: hopefully detection, but no more

%iv. LOFAR/MWA/PAPER 
% Carilli
Three reionization path-finder array experiments are currently
operational: LOFAR, MWA and PAPER. All three are designed to have the
sensitivity to make the first statistical detection of the neutral
IGM. However, the HI line experiment is very challenging due to
foreground continuum emission some four orders of magnitude brighter
than the expected line signal.  The three experiments offer very
significant complementarity, with different intrinsic systematics and
different approaches to foreground mitigation. We emphasize that, for
such an important discovery, multiple approaches are critical in order
to check and verify any claimed (likely low S/N) detection amidst the
substantial systematic uncertainties. The first detection of
the neutral IGM is not the end of reionization studies, just the 
beginning. Recall that some four decades separated the first detection
of the CMB from the first statistical characterization. 


iv. HERA II: 'gauranteed' detection, full characterization, dark ages, imaging
% Aguirre

v. Move 'analysis' stuff here?


\section{Challenges} % 2 pages

\subsection{Foregrounds}  % 1.5pages
% Parsons, Morales

i. relative intensities (Fig: Continuum from MWA)

ii. Details on Delay Spectrum approach

a. key: 3D k-space: show wedge and discuss. chromatic sidelobes with characteristic freq scale 
set by baseline length

b. analysis focuses on line-of-sight PS dimension

c. work in EoR window.  different window levels, depending on effective horizon

d. drives design to redundant array: helps calibration, add spectra coherently

e. dictates geometry of antenna elements and other RF stuff. avoid standing waves of given length. 

f. area: drives sensitivity as used in section 2

g. some words about why compact, hex array 

\subsection{Other Challenges} % 0.5 pages

i. Interference: Karoo RFI plots  FIG 
% Jacobs

i. Polarization 
% Moore, Aguirre

D. Ionosphere: 

i. short baselines and narrowish FoV

ii. direction dependent gains?


\section{HERA design and project} % 8 pages

\subsection{Lessons learned recap} % 0.5 page
The HERA collaboration has made significant progress on multiple approaches in dealing with
foregrounds.
Based on a ``delay-spectrum'' understanding of
the mechanism for how instrumental responses modulate foregrounds on
spectral scales of cosmological interest \citep{parsons_et_al2012b},
PAPER has optimized its instrument to focus on regions in Fourier
space that have weak coupling to foregrounds caused by the
interferometer.  These regions are determined both by chromatic
instrumental responses and by the inherent frequency structure of the
foregrounds.  An `EoR window' has been identified in the Fourier
(wavenumber) space of spectral and angular power spectra that is
inherently free of continuum emission, without explicit continuum
subtraction in either the image or spectral domain \citep{pober_et_al2013,morales_et_al2012,Datta_2010}
This window allows for continuum
`avoidance' rather than subtraction. Observations based on this new
approach have already demonstrated that the extremely stringent level
of foreground suppression needed to access the 21cm signal is largely
in hand (as shown in Figure \ref{fig:pk_k3pk}), with upper limits
that are beginning to rule out cold reionization scenarios.

HERA-331 proposal targets a 331-element array that incorporates
our proven foreground avoidance techniques while improving
dramatically the sensitivity relative to current experiments.  With a
new understanding of how antenna size and separation affect
sensitivity and foreground isolation, it has become evident a revision
of the PAPER antenna design can yield up to 20 times the sensitivity
per element without substantially degrading foreground isolation.
Where PAPER's elements lack collecting area and are smaller than
strictly required for foreground isolation, and the majority of MWA
and LOFAR elements are spaced too widely to avoid foregrounds,
HERA-331 employs an extremely compact array of 14-m parabolic dishes
with PAPER-style dipole feeds (see Figure \ref{fig:hera_dish}.  The
short (4.5m) focal height of these dishes is central to limiting the
path length of reflections whose time-delay gives rise to chromatic
instrumental systematics.

\begin{figure}[!ht]
\centering
	\begin{subfigure}[b]{0.46\textwidth}
		\includegraphics[width=\textwidth]{plots/paper_element.jpg}
		\caption{The PAPER element (provides a clean instrumental response as a function
		of frequency \citep{parsons_et_al2010,parsons_et_al2012b}, which is crucial to
		the foreground isolation shown in Figure \ref{fig:eor_pspec}.}
	\end{subfigure}
	\quad
	\begin{subfigure}[b]{0.46\textwidth}
		\includegraphics[height=1.75in]{plots/hera_dish.png}
		\caption{A 14m dish designed around the feed dramatically improves sensitivity while
		constraining the path length and amplitude of reflections to ensure that foreground 
		isolation is not substantially degraded.}
	\end{subfigure}
\caption{PAPER and HERA elements}
\label{fig:hera_dish}
\end{figure}

The size of HERA-331 dishes optimizes cost for a fixed sensitivity and
level of foreground isolation.  The associated reduction in the number
of antenna elements to achieve a given collecting area, combined with
the fact that these dishes have no moving parts, are built from
inexpensive materials, and follow a simple construction that can be
contracted locally, makes the cost of building HERA-331 substantially
cheaper than was anticipated in the roadmap submitted to \nwnh\ for this
stage of the program.   

HERA leverages the technical heritage of PAPER, MWA and of CASPER\footnote{Collaboration for Astronomical
Signal Processing and Electronics Research, a Berkeley-initiated worldwide open source community that is
developing boards, firmware and software for the astronomical community} and incorporates a
phased implementation to mitigate against risk.  The system, technology and phased approach is discussed below
in more detail.

\vspace{-0.25in}
\subsection{Instrument Design}
\vspace{-6pt}
\label{InstDes}
HERA has a very straightforward insrument design.  The element
itself is a fixed zenith-pointing 14-meter segmented prime-focus paraboloid with a high screen
to minimize cross-talk between elements (which are spaced 14.3 meters on a
hexagonal grid.   The $f/D$ of
the paraboloid is 0.32, so that the focal length, $f$, is less than 5 meters to meet the 
standing wave specification at the delays of interest of more than 60 dB of attenuation at delays 
greater than 15 ns.


The active feed sends back the entire dual-polarization analog bandwidth on standard
coaxial cable to an aggregation point called a ``node'', which services 
about 15 antennas.  This cable length is kept short (35-m) to keep any standing
wave contamination outside of the delay-space of interest for power spectrum
measurements.  The node amplifies, filters, digitizes and transmits the signal data stream
back to the central processing location (the Karoo Array Processing Building - KAPB).
Figure \ref{fig:blockDiagram} shows the system.

\begin{figure}[h]
\centering
\includegraphics[width=\textwidth]{plots/Engineering/HERA_high_level_block_diagram.jpg}
\caption{Block diagram of the HERA system.}
\label{fig:blockDiagram} 
\end{figure}

\vspace{-0.25in}
\subsection{Parabolic Dish Element}
\vspace{-6pt}

The element is a fixed zenith-pointed mount, which dramatically simplifies design and operation.  
Cost is a key design constraint for this experiment, which has components for construction materials, 
assembly in a remote area, and operation over its lifetime.  As an experiment, the design lifetime is 5 years, 
which helps constrain costs relative to a long-lived facility.  For a field-deployed instrument, one needs to 
define limited well-constrained reference points for accuracy in construction.  The proper installation procedure 
is therefore critical in construction.  The elements are also nearly abutting one another, which allows for sharing 
of physical support infrastructure.

Many competing factors set the area of the collecting element.  Among them are
sensitivity, cost, minimum baseline, efficiency and delay values of internal reflections.
Using the derived sensitivity for a redundant array \citep{parsons_et_al2012a} 
and a reasonably complete costing model, one can compute the cost/performance for a 
fixed sensitivity as a function of diameter.  
Figure \ref{fig:nvsd} shows 
this function normalized to a 14-meter antenna, indicating a fairly broad minimum 
extending from about 14 to 22 meters.  Within this range, smaller elements are
preferred because of increased field of view and lower timescales for systematics
arising from reflections. 

\begin{figure}[h]
	\centering
	\begin{subfigure}[b]{0.46\textwidth}
		\includegraphics[width=0.9\textwidth]{plots/Engineering/nvsd.png}
		\caption{Costing model for a fixed sensitivity and varying the diameter.}
		\label{fig:nvsd} 
	\end{subfigure}
	\quad
	\begin{subfigure}[b]{0.46\textwidth}
		\includegraphics[width=0.8\textwidth]{plots/Engineering/focalEff.png}
		\caption{Analytical model efficiency of a parabolic element as a function of focal height and diameter.}
		\label{fig:disheffic}
	\end{subfigure}
	\caption{Cost and performance plots to determine diameter.}
\end{figure}

Given the delay-spectrum technique developed for 21-cm EoR science, it is important 
to ensure that internal reflections within the antenna structure
are at a low enough level at delay values where the array has sensitivity to the EOR power spectrum.  
That is, we don't want reflections that will modulate foregrounds to corrupt scales corresponding to 
the desired $\kpar$, scattering foreground power into the EoR window.  Recent work characterizing 
foregrounds suggests that the spectral structure of foregrounds observed with
baseline separations of $8\lambda \approx 15$m are well-behaved to current limits \citep{parsons_et_al2013}. 
This sets the upper limit for the dish diameter at about
15m.  However, this requirement also sets restrictions on the focal length of
the dish since the primary resonances in the dish will be the standing waves
that arise between the primary reflector and the antenna feed, caused by
imperfect impedance matches of the feed electronics and free space, as well as by the presence of any 
metallic structure in the area.

% FOLLOWING SEEMS TOO DETAILED FOR A PROPOSAL

%To distinguish between measured sky delays and instrumental-induced delays at a required threshold ($R_T$) dB, 
%we need sufficient attenuation of the reflected signals at a given focal length ($f$) at a specified delay ($\tau_{d}$).  Assuming a conservative simple model that the magnitude of the reflection is attenuated by $A$ dB at each reflection 
%we see that for a delay length limit of $\delta_{d}=c\tau_{d}$ and focal length $f$,  the required focal length is
%
%\begin{equation}
%f < \left(\frac{A}{R_T}\right)\delta_d.
%\end{equation}
%
%Using nominal values of $R_T$ =  60dB (an order of
%magnitude below where EoR is predicted to be below foregrounds) at delays
%corresponding to the time it takes travel 15m and a net attenuation of 20 dB per reflection, 
%we find that the focal length should be less than about 5 m.  Free-space loss effects would 
%increase that value, loosening the constraint.
%
To maximize sensitivity, we wish to maximize the efficiency of the
HERA element, while still thresholding the delays at which reflections couple
back into the feeds. The design for the proposed HERA element is currently set
at $14m$-diameter dish with a focal height of $4.5$ m, which currently gives us
good total efficiency (see Fig \ref{fig:disheffic}) while staying within the delay-response constraint. 
Analytical models of the beam pattern are shown in Figure \ref{fig:beam}.
Full electromagnetic modeling will be coupled with 
physical measurements of the prototype to validate and refine the element.

%In addition to the constraints given by the element itself, the HERA element size is
%also influenced by the location of the ``knee'' in the EoR power spectrum \citep{lidz_et_al2008}
%The EoR power spectrum has an upward slope for low $k$-modes which levels off
%around $k=0.15 h$/Mpc %(see fig blah). 
%Working inside this $k$-mode would be beneficial due to the fact foregrounds are
%less problematic. This poses a problem because without knowing the
%width of our foregrounds, we can't say for sure which $k$-modes (in the power
%spectrum) are corrupted. This uncertainty, coupled with increasing systematic affects for shorter 
%baselines (hence smaller diameters), favors larger diameter antennas.

\begin{figure}[h]
	\centering
	\begin{subfigure}[b]{0.46\textwidth}
		\includegraphics[width=0.85\textwidth]{plots/dish.png}
		\caption{CAD model of 14m dish with screening and some supports removed to show detail.}
		\label{fig:dish} 
	\end{subfigure}
\quad
	\begin{subfigure}[b]{0.46\textwidth}
		\includegraphics[width=\textwidth]{plots/Engineering/hera_beam.png}
		\caption{Analytical model beam patterns at 10m, 12m and 14m.}
		\label{fig:beam} 
	\end{subfigure}
	\caption{HERA element and beam pattern.}
\end{figure}

%The $k$-mode for a $14m$ baseline is $k = 0.023$, given by $k_{H} =
%\frac{B}{c}\frac{dk}{d\eta}$, where $B$ is the length of the baseline, 14m in
%our case, $c$ is the speed of light, and $\frac{dk}{d\eta}$ is the cosmological
%transfer function from delays to $k$-modes. In addition, narcissistic
%reflections add into our $k$ budget as well. The $f$ and $D$ choices above provide a $k$ budget for
%foreground widths to be within $\Delta{k}\sim{0.1}$. Testing these hypothesis
%and again, finding a compromise between maximum sensitivity, foreground budget
%will be key to testing and constructing the required element. Foreground width constraints
%are still an area of active research \citep{pober_et_al2013}.

The core defining elements in this design are the central hub and three tall support poles.  Three intermediate 
support posts are installed between each pair of poles.  A 2$^{\prime\prime}$ PVC spar of 24.1$^{\prime}$ 
terminates at each pole and post.  These spars are supported at each end and one point in the middle at the 
proper height and angle.  The intervening PVC pipe essentially acts as a smoothing filter between those 
points, noting also that a beam with point loads attains nearly the quadratic shape desired.  The CAD model is 
shown in Figure \ref{fig:dish}.  

The tall ($\sim$ 7m) poles provide locational accuracy (in all three dimensions) for the overall array installation.  
Using conventional commercial pole-installation techniques the poles are installed first for the entire array.  
Note that every pole except for the edge poles are shared by three antennas.  A standard theodolite can 
then be used to mark a known level height on all three poles.  These locations are then used with tensioned 
lines to define the center of that element and the hub is positioned at that location using a jig.

The hub uses concentric commercially available ``sonotube'' forms (circular cardboard forms for concrete pillars) 
and PVC sleeves to hold the PVC spars and PVC supports.  The retaining holes may be accurately cut into 
the forms, sleeves installed and concrete poured to make a simple hub to the desired accuracy.  The jig holds 
the concentric rings in place and allows it to be centered by tensioned lines while the concrete is poured.  
When the concrete cures one can then transfer an accurate offset from the dish vertex back to the poles.
The intermediate posts are then located by the support sub-assemblies attached to the poles and posts along 
with the rim sub-assemblies.  These are positioned and a small pier is poured to locate them.  After spar support 
pieces are installed on the posts, the spars themselves and the metal cloth can be installed.

The feed is held off the three tall poles using tensioned lines to accurately locate it over the hub.  A precise 
length of kevlar rope holds the feed down to the hub at a precise focal point.  The RF cables follow the line 
down and out to the analog-to-digital converters and correlator.  

To minimize cross-talk, metal screens are strung between every pole/post, which go to the level of the feed.  
The dishes have a rim-to-rim spacing of 30cm to allow the screen to be slightly angled to minimize standing waves.


\vspace{-0.25in}
\subsection{Signal Path}
\vspace{-6pt}

The signal path inherits directly from PAPER's legacy,
including the feed,  low noise amplifier, coaxial cable, and receiver.  The
design of the collecting element was influenced by several criteria. It must deliver (1) a
clean, smooth primary beam pattern in both polarizations with minimal
sidelobe response, (2) a tightly bounded feed-point impedance of
reasonable value, (3) a rugged structure that is physically and electrically stable
over time, (4) manufacturability. For the feed, we use a dual-polarized
version of the sleeved dipole design: a twin-resonance structure
consisting of a pair of crossed dipoles made from copper tubing
located between a pair of thin aluminum disks \citep{parsons_et_al2010} as
shown in Fig. \ref{fig:element}.
%\begin{figure}[!ht]\centering
%
%\end{figure}

\begin{figure}[h]
	\centering
	\begin{subfigure}[b]{0.3\textwidth}
		\includegraphics[width=\textwidth]{plots/new_antenna_closeup.jpg}
		\caption{Photograph of the PAPER antenna positioned above the trough reflector.}
		\label{fig:element}
	\end{subfigure}
	\quad
	\begin{subfigure}[b]{0.3\textwidth}
		\includegraphics[width=\textwidth]{plots/Engineering/recv_node.png}
		\caption{Receiver modules in a node refrigerator.}
		\label{fig:recv_node} 
	\end{subfigure}
	\quad
	\begin{subfigure}[b]{0.3\textwidth}
		\includegraphics[width=0.8\textwidth]{plots/Engineering/digital.png}
		\caption{Digital equipment (ADC/correlator) in container.}
		\label{fig:digital} 
	\end{subfigure}
	\caption{Photographs of existing instrument components to be used.}
\end{figure}

PAPER's front-end amplifier has a low noise figure with
moderate gain, high dynamic range to tolerate RFI, unconditional
stability to ensure oscillation-free operation, well-matched impedance
to both the antenna and cable, well-understood temperature
dependence, a mechanically rugged mechanical design, low
susceptibility to electrostatic discharge, low power consumption, and
low fabrication cost.  It is housed in a metal enclosure affixed to
the antenna to form a very rugged, reliable, low-cost unit with
excellent RF performance \citep{parsons_et_al2010}.

The signal is transported to the central container via relatively
inexpensive RG-6 75 Ohm coaxial cables having a polyethylene jacket.
These cables are not buried and so were chosen to be very rugged,
allowing
PAPER to explore different array configurations by moving antenna elements.
Dual-channel receiver boards (Fig. \ref{fig:recv_node}) add
amplification and a band-limiting RF filter after cable transmission
into the central hut.  Receiver boards are housed in a special
enclosure with RFI shielding to prevent self-interference from the
digital correlator and feedback to the antennas.
HERA-61 also reuses the existing 128-input PAPER correlator (Fig. \ref{fig:digital})
in the existing container.

\vspace{-0.25in}
\subsection{Prototype Construction and Testing}
\vspace{-6pt}

One prototype of the antenna has been constructed near the Radio Astronomy Lab in California. 
This prototype serves as an important first construction test-bed and is currently being used to do 
initial network analyzer measurements (Fig. \ref{fig:heracles}).
This proposal calls for the construction of two additional prototype dishes alongside
the PAPER array deployed at the NRAO site near Green Bank, WV.
These dishes will be tested
with a network analyzer in situ, and will be cross-correlated with PAPER elements using
the correlator currently deployed on site, in order to measure
the element performance and optimize the design.  The goal of this effort is to ensure
that all signal reflections
are attenuated by a factor of -60 dB by the time that they are capable of entering the signal path
at a delays greater than 50 ns.  While signal reflections will be inevitable with such a dish
design, the quality of the impedance match at the feed,
the presence of structures that reduces resonances between
the feed and the dish, and control of the focal height of the parabola are all aspects
of the design that can
be manipulated to help achieve this specification, ensuring that reionization modes above
$k_\parallel=0.1h {\rm Mpc}^{-1}$ are not dominated by foreground contamination.

\begin{figure}[h]
	\centering
	\begin{subfigure}[b]{0.46\textwidth}
		\includegraphics[width=\textwidth]{plots/heracles.png}
		\caption{Picture of the existing construction prototype in California.}
	\end{subfigure}
	\quad
	\begin{subfigure}[b]{0.46\textwidth}
		\includegraphics[width=\textwidth]{plots/Engineering/heraclesNA.png}
		\caption{Initial reflection measurements from the prototype.}
	\end{subfigure}
	\caption{Prototype antenna and initial results.}
	\label{fig:heracles}
\end{figure}

These additional prototypes will be important to finalize the specific construction techniques to be 
used in the final antenna construction contract in South Africa.

\vspace{-0.25in}
\subsection{Array Construction}
\vspace{-6pt}
Construction of HERA-61 consists of five primary steps: 
\begin{enumerate}[noitemsep,nolistsep]
\item site preparation and surveying 
\item pole installation by contracted labor with specialized utility pole equipment 
\item hub placement and height adjustment
\item remainder of construction of each element, following description above, using local labor 
\item moving the feeds and cables from the existing PAPER dipoles to new ground screens by project staff
\end{enumerate}

The contractors and immediate supervisors will be based in South Africa.  Supervision staff 
will be part of the extensive support infrastructure in place at the site to support South African SKA activities.
As mentioned above, the tall poles are shared amongst the antennas in the tight configuration.  
Therefore, 61 antennas requires only 75 poles rather than 61x3=183.  These are standard telephone/power 
utility poles and a great deal of  expertise and infrastructure exists to install these in remote settings.  
Figure \ref{fig:config} shows the configuration of the 61 antennas including the tall pole locations and \ref{fig:optics} shows a cross-section of an element.

\begin{figure}[h]
	\centering
	\begin{subfigure}[b]{0.35\textwidth}
		\includegraphics[width=\textwidth]{plots/Engineering/hex_61.png}
		\caption{Configuration of the 61-element array.  Note the rectangular container to scale on right.}
		\label{fig:config}
	\end{subfigure}
	\quad
	\begin{subfigure}[b]{0.6\textwidth}
		\includegraphics[width=\textwidth]{plots/Engineering/optics.png}
		\caption{Cross-section and optics of an element.}
		\label{fig:optics}
	\end{subfigure}
	\caption{Configuration and optics.}
	\label{fig:config_optics}
\end{figure}

Except for two custom metal assemblies, the remainder of the construction materials are standard wood, 
metal and PVC parts.  The two custom assemblies are simple small welded metal assemblies for the end of 
rim support pieces (quantity 12) and the feed and feed backplane (quantity 1).  The PVC and wood sub-assemblies 
will be constructed off-site where material and labor is readily accessible and shipped to site.  The 
construction of these sub-assemblies, pre-cut wood and PVC, and wire mesh pieces will be done on-site under 
contract.  Project staff will then move over the existing PAPER feeds and cables and commission the array.  

\vspace{-0.25in}
\subsection{Commissioning}
\vspace{-6pt}
Note that since all of the electronics are already deployed and working, commissioning will consist of verifying the existing performance but within the new electromagnetic environment presented by the new ground screens.  Below is
an explicit list of commissioning tasks:
\begin{itemize}[noitemsep,nolistsep]
\item equalization of signal levels and repairing of reflections and misbehaving signal paths due to cabling using auto-correlation data 
\item measure system temperature from raw level of auto-correlation data as it varies diurnally.
\item measure relative width of primary beam using source transits for XX and YY polarizations 
\item use established sky model from PAPER and correlation with known PAPER feeds to 
determine absolute gain as a function of direction for new dishes. 
\item use redundancy to solve for phase and gain calibration parameters, measure stability of parameters versus 
time 
\item detailed characterization of cross-coupling and reflections between antennas using cross-correlation data and imaging 
\item fold calibrated data on the basis of redundancy and multiple observing days to obtain a high-sensitivity delay spectrum capable of verifying absence of low-level reflections at higher delays. 
\end{itemize}

%%%%%%%%%%%%MRI VERSION
%%%%%%%%%%%%%%%%%%%%%%%%%%%%%%%%%%%%%%%%%%%%%%%%%%

\subsubsection{Element and Configuration}
The element diameter of 14 meters is an optimization of cost, systematics-reduction, and the delay-contamination specification.  Figure \ref{fig:diameterOpt} shows the costing curve for a fixed sensitivity (normalized to 14 meters) 
along with arrows indicating directions of increasing systematics and increasing delay-space contamination.  The 14-m
chosen diameter is in the ``allowed'' window near the cost minimum, but biased towards smaller elements to increase
the field-of-view.  Figure \ref{fig:beampattern} shows the beampattern.

\begin{figure}[!ht]
	\centering
	\begin{subfigure}[b]{0.46\textwidth}
		\includegraphics[width=\textwidth]{plots/Engineering/nvsd.png}
		\label{fig:diameterOpt}
		\caption{Cost curve for dish diameter at a fixed sensitivity.  The arrows indicate the regions of 
				increasing systematics and delay-space contamination.}
	\end{subfigure}
	\quad
	\begin{subfigure}[b]{0.46\textwidth}
		\includegraphics[width=\textwidth]{plots/Engineering/focalEff.png}
		\label{fig:focalEff}
		\caption{Overall efficiency for elements of various diameter vs focal length.  The delay 
				contamination specification is $f<5$m.}
	\end{subfigure}
	\label{fig:elementcharacteristics}
	\caption{Variational analysis data for the element.}
\end{figure}

Figure \ref{fig:configuration} shows the layout of the array. 



\subsubsection{Analog System}
ii. 14m elements + broad band (active) dipole feed
% de Boer, Bradley

\subsubsection{Digital System}
The advent of CASPER, initiated by two of the PIs of this proposal, has led to a resurgence of radio telescope 
backend development.  Digital systems used to be the achilles heel of array correlator design, but is now
generally viewed as a solved problem.  CASPER hardware evolves as groups develop hardware, software and
techniques which get shared amongst the community.

iii. Signal transport: RF to nodes, digitize at nodes. fiber to correlator. 
% Werthimer?

iv. Digital stuff: emphasize 'solved problem'. Casper work. block diagrams...
% Parsons

\subsubsection{Data Handling}
C. Post correlator data path
% Aguirre, Moore

i. local storage

ii. transport to data centers

iii. data centers: access? 

\subsection{Analysis}
D. Data analysis [Move up?]

i. Calibration 
% Aguirre, Morales

ii. Continuum and removal
% Morales

iii. Delay spectrum analysis
% Pober

iv. Other (related) approaches: Bayes, z-variance,
% Liu, Carilli

v. HI imaging
% Jacobs

E. Array monitoring/maintenance: daily, weekly health monitoring
% de Boer

F. Aspirations [move to facilities section?]

i. FFT correlator

ii. other

\subsection{Five year Schedule and deliverables} % 0.5 page
% De Boer

i. stages and science

ii. first science within 3 or 4 years

iii. main science results before end of decade


\section{Broader impacts} % 1 page
% Aguirre

A. pre and post-doctoral students

B. South Africa connection

i. professional development/exchanges

ii. broader outreach in SA

C. Broader astronomy community

i. Data access: transients, SETI, ionosphere

ii. Cross correlation: CMBpol, CO/CII IM

D. Casper open source: good example to build from


\section{VI. Summary: why us and why now?} % 0.5 pages
% Parsons

that's obvious


\clearpage
\setcounter{page}{1}
\thispagestyle{empty}
%\bibliographystyle{apj}
%\bibliographystyle{hapj}
\bibliographystyle{jponew}
\bibliography{biblio}

\end{document}

