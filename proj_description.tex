Feb 5, 2014
Draft outline: MSIP
Chris Carilli

I. Overview ~ project summary 
A. We propose HERA: next step in reionizastion roadmap
B. Fulfill NWNH high-priority goals
C. New understanding/techniques => faster, better, cheaper
D. Brief summary of timeline: science along the way, major results before end-decade


II. Science: keep science up-front
A. General intro 
i.physical concepts: reionization and dark ages
ii. Current knowledge: various constrai	nts, 1st galaxy studies...
iii. Important role of 21cm studies highlighted in NWNH
a. Typical ideal sim results: T_B vs. z 'cube' and corresponding power spectrum evolution (Fig)
b. introduce some of the important parameters/processes explored: when? how?
bubble scale? sources? inside out? [much of this could be left for below?)
c. reemphasize that current knowledge is nil, yet demand is high
d. emphasize unique (only?) probe of dark ages

B. HI reionization and dark ages science: examples using HERA 331 fiducial sensitivity
[I recommend assuming a 'medium' sensivity, eg. PB horizon, not horizon horizon, but could show both,
at risk of TMI]

i. F(HI) vs. z: HERA vs. other techniques (Fig) -- emphasize constaints in unexplored territory
ii. PS sensitivity at fixed z: 127, 331 (Fig)
iii. Various covariance analyses: constraints on different physical processes (Liu/Pober analysis). but
please -- keep this down to a couple of incisive figures and fiducial models. wall paper doesn't sell very well. 
iv. Imaging large scale structure: simulation (Fig) 
    a) Imaging as a probe of non-Gaussianity and topology of reionization (Fig from Watkinson & Pritchard)
    [Not sure if this belongs here, should discuss: b) Bayesian imaging (Figs from Paul Sutter).  Perhaps a broader impact on the radio community too?]
v. Dark ages: getting into 'real cosmology' (linear structure evolution)  (Fig)
vi. 21cm forest: perhaps a paragraph on possibilities of small scale structure, if we can find radio galaxy? (Fig) 
vii. Cross-correlation science: perhaps montage Fig showing HI(Tb), galaxies, dark matter... at given epoch
a. Provide environmental context for ALMA/JWST 1st galaxy studies
b. CMB pol
c. CO/CII IM
d. note: xcorr further mitigates continuum systematics

C. Where we are right now: PAPER, MWA (LOFAR?). probably need such a section. not sure where.
i. describe arrays, state design driven by new understandings as delineated below (Fig)
ii. Using new techniqes, current best limits  (Fig)
iii. what will happen in next 2 years: hopefully detection, but no more
iv. LOFAR: not sure what to say about lofar. but we could emphasize this is difficult measurement with 
profound astrophysical impact. such a situation demands multiple approaches, since initial 'discovery' is 
likely to be low S/N and strongly affected by systematics. multiple approaches are crucial check/verify. 
moreover, even if LOFAR beats MWA and PAPER to discovery, that will just be the start of reionization 
studies, not the end. HERA II is that next required step. 
iv. HERA II: 'gauranteed' detection, full characterization, dark ages, imaging


III. Challenges 

A. Foregrounds
i. relative intensities (Fig: Continuum from MWA)
ii. Details on Delay Spectrum approach
a. key: 3D k-space: show wedge and discuss. chromatic sidelobes with characteristic freq scale 
set by baseline length
b. analysis focuses on line-of-sight PS dimension
c. work in EoR window.  different window levels, depending on effective horizon
d. drives design to redundant array: helps calibration, add spectra coherently
e. dictates geometry of antenna elements and other RF stuff. avoid standing waves of given length. 
(lets avoid talking in delays, but use lengths)
f. area: drives sensitivity as used in section 2
g. some words about why compact, hex array 

B. Interference: Karoo RFI plots  FIG

C. Ionosphere: 
i. short baselines and narrowish FoV
ii. direction dependent gains?


IV. HERA design and project: probably longest section. skeleton below

A. reemphasize dictated by new understanding

B. Describe in detail main parts of hardware project
i. compact hex grid array
ii. 14m elements + broad band (active) dipole feed
iii. Signal transport: RF to nodes, digitize at nodes. fiber to correlator. 
iv. Digital stuff: emphasize 'solved problem'. Casper work. block diagrams...

C. Post correlator data path
i. local storage
ii. transport to data centers

D. Data analysis
i. Calibration 
ii. Continuum and removal
iii. Delay spectrum analysis
iv. Other (related) approaches: Bayes, z-variance,
v. HI imaging

E. Array monitoring/maintenance: daily, weekly health monitoring

F. Aspirations
i. FFT correlator
ii. other

G. Five year Schedule
i. stages and science
ii. first science within 3 or 4 years
iii. main science resultsbefore end of decade


V. Broader impacts

A. pre and post-doctoral students

B. South Africa connection
i. professional development/exchanges
ii. broader outreach in SA

C. Broader astronomy community
i. Data access: transients, SETI, ionosphere
ii. Cross correlation: CMBpol, CO/CII IM

D. Casper open source: good example to build from


VI. Summary: why us and why now?

that's obvious










