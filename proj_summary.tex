\documentclass[preprint]{aastex}
\usepackage[top=1in, bottom=1in, left=1in, right=1in]{geometry}
\usepackage{amsmath}
\usepackage{graphicx}
\usepackage{mdwlist}
\usepackage{natbib}
\usepackage{natbibspacing}
\setlength{\bibspacing}{0pt}
\setlength{\parskip}{0pt}
\setlength{\parsep}{0pt}
\setlength{\headsep}{0pt}
\setlength{\topskip}{0pt}
\setlength{\topmargin}{0pt}
\setlength{\topsep}{0pt}
\setlength{\partopsep}{0pt}
\setlength{\footnotesep}{8pt}
\pagestyle{empty}

\def\HI{{H{\small I }}}
\def\HII{{H{\small II }}}

%Project Summary. (1 page maximum) Required elements include an overview of the
%proposed program, and separate entries addressing the intellectual merit and
%broader impacts. The summary should be written in the third person, informative
%to those working in the same or related field(s), and understandable to a
%scientifically or technically literate reader.

\begin{document}
\pagestyle{empty}

\title{HERA: Illuminating Our Early Universe}


%Overview: Insert a self-contained description of the activity that would result if the proposal were funded and include a statement of objectives and methods to be employed. 
We propose to build the Hydrogen Epoch of Reionization Array (HERA) to make precision measurements of our Cosmic Dawn, as the first stars and galaxies burn away the primordial hydrogen fog. HERA is optimized to study \HI 21~cm emission from the primordial intergalactic medium (IGM) throughout the epoch
of cosmic reionization and reaching back to the end of the Dark Ages ($z = 6$ to 30). During these epochs, the first stars and 
black holes warm and reionize the neutral IGM that pervades the Universe following recombination. 
HERA will perform detailed characterization of the evolution of the \HI 21~cm power spectrum and make the first images of large scale features during reionization. 
These results will make constraints on the timing of cosmic reionization, 
the fundamental properties of the first galaxies, the evolution
of large scale structure, and the very early warming of the IGM by the first generation of stars. 

HERA's design draws upon the technical heritage of PAPER, MWA, and MITEoR instruments, and consists of a close-packed hexagonal array of 331 14~m zenith pointing parabolic dishes covering the 50 to 225~MHz band and located at the radio-quiet SKA site in South Africa. The HERA collaboration has pioneered a new technique in interferometric power spectral
analysis based on the three-dimensional nature (spectral and spatial) of the
\HI 21~cm signal. This technique has opened a `reionization window' in power spectral space that is 
largely free of contamination from the strong foreground continuum emission, as demonstrated with PAPER
and MWA observations. HERA is optimized to 
exploit this window and has the sensitivity and precision to observe the \HI power spectrum over a broad range in redshift and to make the first images of reionization. 

%As endorsed in the recent astronomy decadal survey, 
%Hydrogen Epoch of Reionization Arrays (HERA) is a roadmap for characterizing
%cosmic reionization --- the epoch when the first luminous sources ionized the
%bulk of the hydrogen in the Universe --- via redshifted 21-cm hyperfine
%emission from neutral hydrogen.  Following on the successes of first-generation
%HERA instruments (PAPER, MWA, LEDA, MITEoR, and EDGES), we propose to build
%the next phase of HERA using 
%This instrument brings to bear both experimentally validated
%foreground suppression techniques and the collecting area required to open new windows into our early
%universe.  

%Each stage of HERA delivers new science capabilities:
%HERA 127 will be capable of determining
%ionization fraction versus redshift over the bulk of reionization,
%HERA 331 will constrain the size of ionization bubbles to help determine the properties of
%first galaxies, and HERA 568 directly image larger structures during
%reionization.  HERA will also begin to explore the Dark Ages before 
%reionization, when a wealth of astrophysical and cosmological processes
%impacted the temperature of the intergalactic medium.

\vspace{.25 cm}
\noindent
{\bf Intellectual Merit.}
%Intellectual Merit: Describe the potential of the proposed activity to meet the Intellectual Merit criterion
%
Observing our Cosmic Dawn was chosen as one of the three science goals for the decade by the Astronomy and Astrophysics Decadal Survey (New World New Horizons 2011), and HERA was the top ranked radio project (RMS Panel, 2011). Based on knowledge gained with PAPER and the MWA, we are now ready to build HERA---a telescope
capable of performing detailed characterization of our Cosmic Dawn at
a cost substantially below that originally envisioned in New Worlds New Horizons. 

In addition to directly observing the process of reionization, HERA's measurements will set the context for high redshift observations with JWST, ALMA, TMT and other telescopes. Knowing whether a galaxy observed by JWST or ALMA is in a region that reionized early (center of large \HII bubble), just reionized (bubble edge), or is forming from pristine gas (neutral region) greatly enhances the science return.

%The evolution of the \HI 21~cm signal from the primordial IGM
%depends on myriad physical processes, including:  the expansion of
%the universe, the ignition of the first stars and galaxies, the formation of
%the first massive black holes, and the relative
%velocity of baryonic matter and dark-matter halos.
%It is widely recognized that 
%the 21-cm hyperfine transition is potentially the most
%powerful method with which to probe the evolution of large-scale structure from the dark ages through
%reionization.  
%Study of the first stars and black holes during cosmic reionization has been highlighted as one
%of the primary {\sl discovery areas} of modern astrophysics in NWNH. Observation of the
%first galaxies is a primary science driver for essentially all modern, large area telescopes,
%such as ALMA, JWST, and TMT.  


%Detecting this signal will have a profound impact on our understanding of the earliest
%galaxy formation, the structure of the IGM, and cosmology.
%
%Progress has been made on constraining the evolution of the primordial IGM through
%observations of eg. Lya galaxy demographics and CMB polarization. However,  
%current constraints on cosmic reionization remain rudimentary, and profound questions remain: 
%When did it occur, and over what timescale?  What objects dominated the radiation field?
%How were the objects distributed? What were the most important feedback mechanisms in the transition
%from the first stars to first galaxies, and how did they affect these populations?
%Without new constraints, further progress on theoretical modeling of first galaxy formation and cosmic
%reionization remains problematic.  



%HERA-II will be built in two stages, 
%allowing for technical improvements and yielding fundamental science results throughout:
%
%% following is just repeat from project description. might be overkill, but we need some 
%% specifics in this section.
%
%\begin{itemize}%[noitemsep,nolistsep]
%%
%\item{HERA~127: A 127 element hexagonal array will be deployed by
%the end of proposal year 2. HERA-127 will measure the rise and fall of the
%21~cm reionization power spectrum from $z = 7$ to 12. These results will set tight
%constrains on the evolution of the IGM neutral fraction, and hence determine
%the timing and duration of reionization, in particular during epochs that are difficult or impossible to
%probe with other techniques.}
%%
%\item{HERA~331: A 331 element hexagonal array will be deployed by the end of the third year. HERA-331
%will measure fluctuations in the 21~cm signal over a variety of spatial
%scales to determine the nature and distribution of the first galaxies
%that dominate cosmic reionization. HERA~331 will cover a frequency range adequate to extend precision
%power-spectrum observations back to the end of the 'Dark Ages' ($z \sim 20$),
%when the first stars and black holes warm the primordial IGM. HERA-331 will also have the
%sensitivity adequate to image the largest structures in the primordial
%IGM --- a task previously only considered possible with third-generation
%instruments.}
%%
%\end{itemize}

\vspace{.25 cm}
\noindent
{\bf Broader Impacts.}
%Broader Impacts: Describe the potential of the proposed activity to meet the Broader Impacts criterion
The HERA team has a deep commitment to reaching the broader community and training the next generation of astrophysics experimentalists. Our education and outreach efforts range from internships, REUs, and community college transfer programs
to radio astronomy YouTube videos and bestselling books, to outreach through tribal centers and science museums. 
As part of this proposal we will create a student exchange program where we will host 3--4 South African students per summer at a rotating host institution, with the goal of giving the students the cultural and research experience needed to apply to US graduate programs. 
%Simultaneously our students collaboration with South African scientists will prepare them for future positions in the increasingly international field of radio cosmology.

We will also publicly release the calibrated and foreground-subtracted HERA data, including high-cadence continuum images for transient and multi-band variability studies, deep wide-field surveys, 
and foreground-subtracted image cubes for correlation with other high redshift observations.

%The HERA program will train new instrumentalists at the graduate and undergraduate levels, involving 
%students in all stages of HERA development and observation. We also fund an
%undergraduate specialist position at UC Berkeley's RAL, offered annually,
%mentoring the individual in the skills required to pursue graduate
%research in instrumentation.
%
%An important HERA activity is the cooperative education of
%under-served students from South Africa in STEM fields.
%We have established formal collaborations
%with faculty at South African universities to engage doctoral students in the HERA project.
%We will establish student exchanges to enhance
%the diversity of US graduate programs by preparing South African students for
%admission to US degree programs.
%Involvement in the program will help ensure that these students are well-positioned for
%future success.
%
%HERA will disseminate calibrated data products to the community,
%including high-cadence continuum images, deep wide-field continuum images, 
%and foreground-subtracted line image cubes.  These enable
%auxiliary science, including cross-correlation analyses with other 
%complementary large-scale probes of reionization. The scientific results from HERA-II will inform the 
%scientific and technical development of the full SKA-low sometime in the next decade.

% might leave out last sentence?

\end{document}
