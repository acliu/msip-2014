\documentclass[preprint]{aastex}
\usepackage[top=1in, bottom=1in, left=1in, right=1in]{geometry}
\usepackage{amsmath}
\usepackage{graphicx}
\usepackage{mdwlist}
\usepackage{natbib}
\usepackage{natbibspacing}
\setlength{\bibspacing}{0pt}
\setlength{\parskip}{0pt}
\setlength{\parsep}{0pt}
\setlength{\headsep}{0pt}
\setlength{\topskip}{0pt}
\setlength{\topmargin}{0pt}
\setlength{\topsep}{0pt}
\setlength{\partopsep}{0pt}
\setlength{\footnotesep}{8pt}
\pagestyle{empty}

%Project Summary. (1 page maximum) Required elements include an overview of the
%proposed program, and separate entries addressing the intellectual merit and
%broader impacts. The summary should be written in the third person, informative
%to those working in the same or related field(s), and understandable to a
%scientifically or technically literate reader.

\begin{document}
\pagestyle{empty}

\title{HERA: Illuminating Our Early Universe}

\section*{Overview}
%Overview: Insert a self-contained description of the activity that would result if the proposal were funded and include a statement of objectives and methods to be employed. 

We propose to build the next phase of the Hydrogen Epoch of Reionizatin Array (HERA-II). HERA-II will
be a close-packed hexagonal array of 331 14-m fixed-pointing parabolic dishes, observing from 50 to 225 MHz,
situated in the radio-quiet zone of the Karoo SKA site in South Africa.
The design of HERA-II is optimized to study the HI 21cm emission from the primordial intergalactic medium through the epoch
of cosmic reionization back to the end of the Dark Ages ($z = 6$ to 30). During these epochs, the first stars and 
black holes warm and reionize the neutral IGM that pervades the Universe following recombination. 
HERA-II will perform detailed characterization of the evolution of the HI 21cm power spectrum. 
The results will provide accurate constraints on the timing of cosmic reionzation, 
fundamental propeties of the first galaxies, the evolution
of large scale structure, and the very early warming of the IGM by the first generation of stars (or more exoctic
phenomena). 

The HERA-II design is driven by reionization science, and builds from the legacy of the HERA-I precusors:
the PAPER, MWA, and MITEoR arrays. The HERA collaboration has pioneered a new technique in power spectral
analysis using interferometers, based on the three-dimesional nature (spectral and spatial), of the
HI 21cm signal. This technique has opened a 'reionization window' in power spectral space that is 
largely free of contamination by the strong foreground continuum emission, as demonstrated with PAPER
and MWA observations. The HERA-II design is optimized to 
exploit this window, and will have the sensitivity to perform high S/N studies of the evolution of 
the HI 21cm power spectrum over a broad range in redshift, while avoiding the strong foreground 
contamination.  However, we emphasize that the HERA-II design is general enough to allow for multiple 
approaches to reionization studies, such as direct imaging, as our knowledge of data calibration and
analysis techniques improves. 

%As endorsed in the recent astronomy decadal survey, 
%Hydrogen Epoch of Reionization Arrays (HERA) is a roadmap for characterizing
%cosmic reionization --- the epoch when the first luminous sources ionized the
%bulk of the hydrogen in the Universe --- via redshifted 21-cm hyperfine
%emission from neutral hydrogen.  Following on the successes of first-generation
%HERA instruments (PAPER, MWA, LEDA, MITEoR, and EDGES), we propose to build
%the next phase of HERA using 
%This instrument brings to bear both experimentally validated
%foreground suppression techniques and the collecting area required to open new windows into our early
%universe.  

%Each stage of HERA delivers new science capabilities:
%HERA 127 will be capable of determining
%ionization fraction versus redshift over the bulk of reionization,
%HERA 331 will constrain the size of ionization bubbles to help determine the properties of
%first galaxies, and HERA 568 directly image larger structures during
%reionization.  HERA will also begin to explore the Dark Ages before 
%reionization, when a wealth of astrophysical and cosmological processes
%impacted the temperature of the intergalactic medium.

\section*{Intellectual Merit}
%Intellectual Merit: Describe the potential of the proposed activity to meet the Intellectual Merit criterion

Study of the first stars and black holes during cosmic reionization has been highlighted as one
of the primary {\sl discovery areas} of modern astrophysics in NWNH. Observation of the
first galaxies is a primary science driver for essentially all modern, large area telescopes,
such as ALMA, JWST, and TMT.  It is widely recognized that 
the 21-cm hyperfine transition is potentially the most
powerful method with which to probe the evolution of large-scale structure from the dark ages through
reionization.  The evolution of the HI 21cm signal from the primordial IGM
depends on myriad physical processes, including:  the expansion of
the universe, the ignition of the first stars and galaxies, the formation of
the first massive black holes, and the relative
velocity of baryonic matter and dark-matter halos.
Detecting this signal will have a profound impact on our understanding of the earliest
galaxy formation, the structure of the IGM, and cosmology.

Progress has been made on constraining the evolution of the primordial IGM through
observations of eg. Lya galaxy demographics and CMB polarization. However,  
current constraints on cosmic reionization remain rudimentary, and profound questions remain: 
When did it occur, and over what timescale?  What objects dominated the radiation field?
How were the objects distributed? What were the most important feedback mechanisms in the transition
from the first stars to first galaxies, and how did they affect these populations?
Without new constraints, further progress on theoretical modeling of first galaxy formation and cosmic
reionization remains problematic.  

Based on knowledge gained with PAPER and the MWA, we are now ready to build HERA-II, a device
capable of performing detailed characterization of the evolution of the primordial IGM at
a cost substantially below that originally envision in NWNH. HERA-II will be built in two stages, 
allowing for technical improvements and yielding fundamental science results throughout:

HERA 127: give science goals and timing. emphasize how we can probe FHI over a z range that is
difficult with other technique

HERA 331: give science goals and timing 

\section*{Broader Impacts}
%Broader Impacts: Describe the potential of the proposed activity to meet the Broader Impacts criterion
% Have not touched this section CC. 

The HERA program will train new instrumentalists at the graduate and undergraduate levels, increase the
diversity of US graduate programs by engaging South African students and
preparing them for admission to US degree programs, and make our major data
products available publicly. HERA will involve graduate students in all
stages of HERA development and observation. We also fund an
undergraduate specialist position
at UC Berkeley's RAL, offered annually,
mentoring the individual in the skills required to pursue graduate
research in instrumentation.

Another important HERA activity is the cooperative education of
under-served students from South Africa in STEM fields.
We have established formal collaborations
with faculty at South African universities to engage doctoral students in the HERA project.
We will establish student exchanges to enhance
the diversity of US graduate programs by preparing South African students for
admission to US degree programs.
Involvement in the program will help ensure that these students are well-positioned for
future success.

Finally, HERA will disseminate various data products to the community,
including wide-field sky maps, high-speed continuum images, deep
foreground-subtracted images, and full-sensitivity data products,  These enable
auxiliary science, including cross-correlation analyses with other 
complementary large-scale
probes of reionization.

\end{document}
